\documentclass[lettersize,journal]{IEEEtran}
\usepackage{amsmath,amsfonts}
\usepackage{algorithmic}
\usepackage{algorithm}
\usepackage{array}
\usepackage[caption=false,font=normalsize,labelfont=sf,textfont=sf]{subfig}
\usepackage{textcomp}
\usepackage{stfloats}
\usepackage{url}
\usepackage{verbatim}
\usepackage{graphicx}
\usepackage{cite}
\usepackage{overpic}
\usepackage{xcolor}
\usepackage{hyperref}
\hypersetup{
    colorlinks=true,
    linkcolor=blue,
    filecolor=blue,      
    urlcolor=blue,
    citecolor=black,
    pdftitle={Overleaf Example},
    pdfpagemode=FullScreen,
    }
    
\usepackage[normalem]{ulem}
\hyphenation{op-tical net-works semi-conduc-tor IEEE-Xplore}
% updated with editorial comments 8/9/2021
\usepackage{amsmath, amssymb, bm}

\newcount\posveccount
\newcommand*\posvec[1]{
        \global\posveccount#1
        [
        \posvecnext
}
\def\posvecnext#1{
        #1
        \global\advance\posveccount-1
        \ifnum\posveccount>0
                ,\
                \expandafter\posvecnext
        \else
                ]^{\mathrm{T}}
        \fi
}

\newcount\colveccount
\newcommand*\colvec[1]{
        \global\colveccount#1
        \begin{bmatrix}
        \colvecnext
}
\def\colvecnext#1{
        #1
        \global\advance\colveccount-1
        \ifnum\colveccount>0
                \\[5pt]
                \expandafter\colvecnext
        \else
                \end{bmatrix}
        \fi
}


\newcommand{\dint}{\int\!\!\!\!\!\int}
\newcommand{\tint}{\int\!\!\!\!\int\!\!\!\!\int}
\newcommand{\qint}{\int\!\!\!\!\int\!\!\!\!\int\!\!\!\!\int}
\newcommand{\td}{\mathrm{d}}
\newcommand{\te}{\mathrm{e}}
\newcommand{\ti}{\mathrm{j}}
\newcommand{\sinfi}{\sin\varphi}
\newcommand{\cosfi}{\cos\varphi}
\newcommand{\sinteta}{\sin\theta}
\newcommand{\costeta}{\cos\theta}
\newcommand{\yref}{y_{\mathrm{ref}}}
\newcommand{\ys}{y_{\mathrm{s}}}
\newcommand{\dref}{d_{\mathrm{ref}}}
\newcommand{\vx}{\mathbf{x}}
\newcommand{\vxo}{\mathbf{x}_0}
\newcommand{\vxc}{\mathbf{x}_{\mathrm{c}}}
\newcommand{\vxs}{\mathbf{x}_{\mathrm{s}}}
\newcommand{\vxr}{\mathbf{x}_{\mathrm{ref}}}
\newcommand{\vk}{\mathbf{k}}
\newcommand{\nn}{\mathrm{n}}
\newcommand{\vn}{\mathbf{n}}
\newcommand{\vnin}{\mathbf{n}_{\mathrm{in}}}
\newcommand{\vhk}{\hat{\mathbf{k}}}
\newcommand{\kn}{k_\mathrm{n}}
\newcommand{\hkn}{\hat{k}_\mathrm{n}}
\newcommand{\hks}{\hat{k}_\mathrm{s}}
\newcommand{\vxref}{\mathbf{x}_{\mathrm{ref}}}
\newcommand{\phix}{\phi'_{x}}
\newcommand{\phixx}{\phi''_{xx}}

\newcommand{\rp}{r_{{\tiny \mathrm{P}}}}
\newcommand{\rg}{r_{\mathrm{G}}}
\newcommand{\rref}{r_{\mathrm{ref}}}

\newcommand{\phiy}{\phi'_{y}}
\newcommand{\phiyy}{\phi''_{yy}}

\newcommand{\phiz}{\phi'_{z}}
\newcommand{\phizz}{\phi''_{zz}}

\newcommand{\phiPxx}{\phi^{P''}_{xx}}
\newcommand{\phiGxx}{\phi^{G''}_{xx}}

\newcommand{\phiPx}{\phi^{P'}_{x}}
\newcommand{\phiGx}{\phi^{G'}_{x}}

\newcommand{\phiPyy}{\phi^{P''}_{yy}}
\newcommand{\phiGyy}{\phi^{G''}_{yy}}

\newcommand{\phiPzz}{\phi^{P''}_{zz}}
\newcommand{\phiGzz}{\phi^{G''}_{zz}}  
\newcommand{\Psynth}{P_{\mathrm{synth}}}  
\newcommand{\Pdiffr}{P_{\mathrm{diffr}}}  
\newcommand{\Pfoc}{P_{\mathrm{foc}}}  
\newcommand{\Pfocsynth}{P_{\mathrm{synth,foc}}}  
\newcommand{\Rssd}{r_{\mathrm{SSD}}}  
\newcommand{\Leff}{\hat{L}_{\mathrm{eff}}} 



\newcommand{\mH}{\mathbf{H}}



\usepackage{amssymb}

\let\oldemptyset\emptyset
\let\emptyset\varnothing

\begin{document}

\title{Limitations of Wave Field Synthesis Part II:\\low-frequency limits}

\author{Gergely Firtha
\thanks{Manuscript received April 19, 2021; revised August 16, 2021.}}


% The paper headers
\markboth{Journal of \LaTeX\ Class Files,~Vol.~14, No.~8, August~2021}%
{Shell \MakeLowercase{\textit{et al.}}: A Sample Article Using IEEEtran.cls for IEEE Journals}

\IEEEpubid{0000--0000/00\$00.00~\copyright~2021 IEEE}
% Remember, if you use this you must call \IEEEpubidadjcol in the second
% column for its text to clear the IEEEpubid mark.

\maketitle
% === Abstract ===

\begin{abstract}
This paper establishes explicit low-frequency validity limits for 2.5D Wave Field Synthesis (WFS) derived from the Kirchhoff approximation. 
We show that the breakdown at low frequencies is governed by the applicability of the stationary phase approximation to the 2.5D Kirchhoff integral.
This is quantified via an effective window (aperture) length that compresses the influence of secondary loudspeaker array size, tapering, referencing, and source/receiver proximity into a single quantity. 
We derive closed-form cutoff conditions for arbitrary source models and listening positions, yielding a cutoff wavenumber that scales with the inverse square of the effective length and recovers classical Fresnel-number/zone theory in frontal directions.
The analysis predicts markedly higher cutoff for lateral positions and for sources/receivers near the SSD, and quantifies the impact of tapering and referencing schemes. 
Simulations with linear and circular loudspeaker arrays for plane, point, and focused sources validate the theory, including the intermediate transition band bounded by the two asymmetric effective lengths. 
\end{abstract}



\begin{IEEEkeywords}
Article submission, IEEE, IEEEtran, journal, \LaTeX, paper, template, typesetting.
\end{IEEEkeywords}

\section{Introduction}

Wave Field Synthesis (WFS) is an advanced spatial audio rendering technique capable of accurately synthesizing sound fields by employing a dense array of secondary loudspeakers distributed along a contour or surface, known as the secondary source distribution (SSD).

Originally introduced by Berkhout in the late 1980s \cite{Berkhout1988, doi:10.1121/1.404755, Berkhout1993:Acoustic_control_by_WFS}, WFS utilizes the Huygens–Fresnel principle to reconstruct acoustic wavefronts over an extended listening area. 
Over subsequent decades, significant theoretical advancements have refined its mathematical framework \cite{Vogel1993:phd, Start1997:phd, Verheijen1997:phd, Sonke1998}, expanded its scope to include innovative virtual source paradigms such as focused and moving sources \cite{Spors2009focused, Wierstorf2013focused, Oldfield2013_FocusedWFS, Ahrens2008moving, doi:10.1121/1.4996126}, and systematically addressed various implementation constraints.

Despite these advances, WFS inherently possesses limitations, particularly at the lower end of its operational frequency range. 
The integral representations used—specifically the Kirchhoff approximation—are fundamentally high-frequency approximations. 
Consequently, low-frequency performance deteriorates below a geometry-dependent threshold \cite{Spors10ahrens:analysis}. 
Although this phenomenon is acknowledged within the literature, a comprehensive analytical treatment and rigorous understanding of the precise low-frequency limits remain largely unexplored.

This paper, as the second part of the series exploring limitations of WFS, specifically addresses these gaps by rigorously examining the low-frequency behavior and limits inherent to WFS. 
The paper builds upon and significantly extends the foundational work presented in the author's doctoral thesis \cite{Firtha2019phd} and utilizes analytical results presented in the first part of the series \cite{Firtha2025_ieee_a}.
The thesis established the general framework connecting ray theory and WFS, including referencing concepts, while \cite{Firtha2025_ieee_a} provided a novel, detailed asymptotic analysis specifically focused on diffraction artifacts and amplitude errors.
The present paper explores the low-frequency limits of the above theoretical framework.
We analyze how the intrinsic approximations influence low-frequency wavefront reproduction and derive explicit criteria and characterizations for assessing low-frequency performance. 
Moreover, we extend our discussion to focused sources, a case exhibiting pronounced low-frequency artifacts due to truncation and tapering effects, areas previously under-examined in existing literature.

% === “Paper structured as follows” paragraph (insert at end of Introduction) ===
The remainder of the paper is organized as follows. Section~II revisits the theoretical bases: the problem formulation, the 3D Kirchhoff approximation, the stationary phase approximation, the derivation of the 2.5D Kirchhoff approximation, and the resulting 2.5D WFS driving functions. 
Section~III develops the low-frequency limits: a qualitative near–far field analysis, the connection to Fresnel-zone theory, and a general validity condition culminating in the closed-form cutoff expression (Eq.~\eqref{eq:low_freq_limit}). 
Section~IV presents application studies: cutoff estimates for linear and circular SSDs with plane, point, and focused sources; spatial maps versus receiver and source position; and the effects of tapering and referencing. 
Finally, Section~V concludes with design guidelines and implications.

Following the principles of reproducible research, the MATLAB code for the figures in this article is publicly accessible at \href{https://github.com/gfirtha/ieee2025b_firtha/tree/main/matlab}{\uline{\small{https://github.com/gfirtha/ieee2025b\_firtha/tree/main/matlab}}}.
Moreover, an open-source toolbox is available at \href{https://github.com/gfirtha/binauralizer}{\uline{\small{https://github.com/gfirtha/binauralizer}}} \cite{firtha_g_2026_18474780} for real-time binauralization of virtual WFS systems with adjustable source and receiver position/orientation, enabling listening tests of the present results, highlighting the lower cutoff frequency, effects of tapering, and supporting alternative WFS referencing schemes.


\section{Theoretical basics}

\subsection{Problem formulation}
The theoretical framework presented in this section largely follows the generalized WFS theory developed in \cite{Firtha2019phd}, adapted here for the specific analysis of low frequency approximations.
For more detailed explanation of the involved quantites the reader should refer to \cite{Firtha2025_ieee_a}.

Consider an arbitrary three-dimensional steady-state sound field expressed generally in polar form as
\begin{equation}
P(\vx,\omega) = A^P(\vx) \, \te^{\ti k \phi^P(\vx)},
\label{eq:general_field}
\end{equation}
with $\vx = \posvec{3}{x}{y}{z}$ being an arbitrary 3D position, $k = \frac{\omega}{c}$ being the acoustic wavenumber, $c$ being the speed of sound and $\omega$ is the temporal angular frequency.
The sound field is characterized by a real-valued, non-negative amplitude function $A^P(\vx) \ge 0$ and a real-valued phase function $\phi^P(\vx)$, which is defined to be non-positive for a diverging field and positive for a converging, focused sound field.
The wavefronts are surfaces characterized by constant phase (isophase, or isochronous in the time domain) surfaces, given by $\phi^P = \mathrm{constant}$. 
These can be locally described up its second order Taylor's approximation by their propagation (ray) direction and curvature when viewed within an intrinsic coordinate system aligned with the ray direction:
\begin{itemize}
  \item We define the normalized local wavenumber vector as
\begin{equation}
\vhk^P(\vx) = -\nabla_{\vx} \phi^P(\vx), \hspace{5mm} |\vhk^P(\vx)| = 1,
\label{eq:eikonal}
\end{equation}
with with $\vhk^P(\vx) = \posvec{3}{\hat{k}_x^P(\vx)}{\hat{k}_y^P(\vx)}{\hat{k}_z^P(\vx)}$ and $\nabla_{\vx}$ denoting the gradient operator in Cartesian coordinates. 
For simple sound fields (point/line sources and plane waves) the local wavenumber vector is a unit vector oriented towards the direction of maximum phase progression, thus indicating the local direction of propagation. 
\item The curvature of the wavefront is described by the second derivative of the phase as
\begin{equation}
\kappa^P_{1,2}(\vx) = \frac{1}{\rho^P_{1,2}(\vx)} = - \mathrm{eigs}\left( \frac{\partial^2 \phi^P(\vx)}{\partial \vx_i \partial \vx_j} \right),
\end{equation}
where $\rho^P_{1,2}$ are the corresponding principal radii of the wavefront \cite{Hartmann1999, Hartmann2001} and with $ \mathrm{eigs}\left( \right)$ denoting the two eigenvalues of the phase function's Hessian \cite{Firtha2025_ieee_a}.
\end{itemize}
In polar form, the gradient of the sound field is expressed by the chain rule as
\begin{multline}
\nabla_{\vx} P(\vx) = \left(\frac{\nabla_{\vx} A^P(\vx)}{A^P(\vx)} + \ti k \nabla_{\vx} \phi^P(\vx) \right)P(\vx) = \\ =
\left(\overline{\kappa}^P(\vx) + \ti k \right) \vhk^P(\vx) P(\vx),
\label{eq:gradient}
\end{multline}
where $\overline{\kappa}^P(\vx) = (\kappa_1^P(\vx)+\kappa_2^P(\vx))/2$ signifies the mean curvature of the wavefront \footnote{Identity $\frac{\nabla_{\vx} A^P(\vx)}{A^P(\vx)} = \overline{\kappa}^P(\vx) \, \vhk^P(\vx)$ is, again, strictly valid for simple sound fields for which case the direction of largest amplitude change coincides with the local propagation direction (c.f. \cite[Eq. (4)]{Firtha2025_ieee_a}).}.
In the far field of the sound source $|\overline{\kappa}^P(\vx)| \ll k$ holds and the gradient can be approximated as
\begin{equation}
\nabla_{\vx} P(\vx) \approx  \ti k \vhk^P(\vx) P(\vx),
\label{eq:pw_appr}
\end{equation}
coined here as a local plane wave approximation, since the equation holds with equality for the case of a plane wave ($\overline{\kappa}^P(\vx) = 0$).

In sound field synthesis, the goal is to replicate this specified \emph{target wavefront} using a continuous array of secondary point sources arranged along a contour $C$, forming the SSD.
These  secondary sources are driven by \emph{driving functions} $D(\vxo,\omega)$, leading to the \emph{synthesized sound field}:
\begin{equation}
\Psynth(\vx,\omega) = \oint_C D(\vxo, \omega) G(\vx-\vxo) \td s(\vxo),
\label{eq:synth_field}
\end{equation}
intended to precisely match the \emph{target field} $P$.
Here, $G(\vx-\vxo) = \frac{1}{4\pi}\frac{\te^{-\ti k|\vx-\vxo|}}{|\vx-\vxo|}$ is the 3D free-field Green's function, describing the radiation from a secondary point source, and $s(\vxo)$ denotes the arc length along the SSD.

The contour of these secondary sources is positioned in a horizontal plane, defined at $z=0$, aligned with the listener's ear level, known as the synthesis plane. 
The associated geometry is depicted in Figure \ref{Fig:geometry}.


\begin{figure} 
  \begin{center}
    \begin{overpic}[width = 1\columnwidth ]{figs/WFS_geometry_2b.png}
	\scriptsize
	\put(61.5,40.5){$\vxo$}
	\put(44,40){$\vxref(\vxo)$}
	\put(24,35){$\vx$}
	\put(3,19){synthesis plane}
	\put(40,22){$C$: SSD contour}
	\put(8,50){$S$: integral surface}
	\put(80.5,71){reference curve}
	\put(80.5,61.5){active SSD}
	\put(56,34){$\vhk^P(\vxo)$}
	\put(84,47){$x$}
	\put(95,29){$y$}
	\put(97,58){$z$}
	\put(81,42){$\vxs$}
	%\put(72,37){$\rho^P(\vxo)$}
  \put(80,66){{\fontsize{7pt}{\baselineskip}
  \selectfont \parbox{4cm}{vertical integral\\path}  }}
	\end{overpic}
\end{center}
\caption{Geometry for deriving 2.5D Kirchhoff approximation and WFS driving functions.}
	\label{Fig:geometry}
\end{figure}

\subsection{The 3D Kirchhoff-approximation}
The starting point of WFS theory involving arbitrary SSD geometries is the simple source formulation (SSF) of an arbitrary target sound field inside a listening region, expressing field $P$ in term of a single layer potential.
For a closed surface $S$ enclosing the listening region the SSF reads \cite{colton1983integral,ColtonKress1998}
{\fontsize{10pt}{\baselineskip}
\selectfont
\begin{equation}
  \small
  P(\vx) = \oint_S  \underbrace{\left(P'_{\mathrm{n}}(\vxo, \omega) - P'_{\mathrm{scat,n}}(\vxo, \omega)\right)}_{\mu(\vxo)} G(\vx-\vxo) \td S(\vxo),
  \label{eq:KA3D}
\end{equation}} 
where $P'_{\mathrm{n}}(\vxo, \omega) = \nabla_{\vx} P(\vx, \omega)|_{\vx = \vxo} \cdot \vnin(\vxo) $ denotes the spatial derivative of the sound field $P$ along the inward normal direction $\vnin(\vxo)$.
The term $P_{\mathrm{scat}}$ is the \emph{exterior} pressure field produced when an acoustically soft surface is placed on~$S$ and excited by $P$; see \cite{Fazi2013:Equivalent_scattering}.  

In the context of 3D WFS the jump in the pressure gradient $\mu(\mathbf{x}_{0},\omega)$ constitutes the driving signal for the secondary sources (c.f. \eqref{eq:synth_field}).
%In the context of WFS the surface of integration $S$ is chosen to be translationally invariant in the vertical direction; its intersection with the horizontal plane is the actual SSD contour (Fig.~\ref{Fig:geometry}). 
%The integration surface is illustrated in Fig.~\ref{Fig:geometry}.
%In the kernel of the integral $P_{\mathrm{scat}}(\vxo, \omega)$ is an exterior field, scattered from a sound soft surface along $S$, on which the scattered pressure is given by $P(\vxo, \omega), \vxo \in S$ \cite{Fazi2013:Equivalent_scattering}. 
%In the integral $\mu(\vxo)$ represents the pressure gradient jump between the interior and exterior sound fields, yielding driving functions for the general synthesis problem.
%The equivalent scattered field is known analytically only in special (planar, spherical, etc.) geometries, while for arbitrary geometries numerical solution is required by using e.g. the Boundary Element Method.
Unfortunately, except for canonical geometries (planar, spherical, cylinderical SSDs) $P'_{\mathrm{scat,n}}$ is not available in closed form, and numerical methods are required—--clearly unsuitable for real-time reproduction---.
%The practical applicability of the above formulation is, therefore, limited in the context of sound field reproduction.

Assuming that the integral/scatterer surface can be considered locally planar within the scale of the wavelength the scattered field can be approximated by $P'_{\mathrm{scat,n}}(\vxo, \omega) \approx -P'_{\mathrm{n}}(\vxo, \omega)$, yielding the Kirchhoff approximation (KA) or tangent plane approximation of the interior sound field \cite{Pierce1991, Ogilvy1991, FelsenMarcuvitz1994, Chew1995}
{\fontsize{9pt}{\baselineskip}
\selectfont
\begin{equation}
  \small
  P(\vx) \approx \oint_S 2 w(\vxo) \, P'_{\mathrm{n}}(\vxo, \omega) G(\vx-\vxo) \td S(\vxo).
  \label{eq:KA}
\end{equation}}
The window function, given by
\begin{equation}
  w(\vxo) = 
  \begin{cases}
    1  \hspace{5mm} \text{if} \hspace{1cm} \hkn^P(\vxo) = \vhk^P(\vxo) \cdot \vnin(\vxo)\geq 0, \\
    0   \hspace{5mm} \text{otherwise}
  \end{cases}
  \label{eq:win}
\end{equation}
performs the partitioning of the integral surface to an acoustically illuminated and a shadow region, denoted by pale red surface in Figure \ref{Fig:geometry}.
For an infinite planar surface the above expression holds with equality, termed as the Rayleigh integral.

The Kirchhoff approximation is fundamentally a high-frequency model that neglects non-local wave phenomena. 
Specifically, it disregards creeping waves due to the local tangent-plane assumption and ignores secondary reflections (multiple scattering) between different parts of $S$ by assuming each surface element radiates into an open half-space.
%The Kirchhoff approximation is intrinsically a \emph{high-frequency} model: creeping waves (diffraction) and secondary reflections between different parts of~$S$ are disregarded.  
The general validity of the KA is a complex, extensively researched question, however, with no general analytical formulation available, so far.
The question frequently arises in the aspect of calculating fields scattered from rough, typically non-convex surfaces, in which typically far-field assumptions are made e.g. scattering of plane waves evaluated far from the scatterer surface.
Sufficient validity conditions, normally derived for plane-wave scattering from rough or non-convex surfaces, are \cite{Franco2017,Fung1981,Eaton2006}
\begin{equation}
      2 k \hkn^{P}(\vxo) \overline{\rho}_{S}(\vxo) \gg 1, \text{or} \hspace{5mm} 2 k \left(\hkn^{P}(\vxo)\right)^3 \overline{\rho}_{\mathrm{S}}(\vxo) \gg 1,
  \end{equation}
with $\overline{\rho}_{S}$ being the mean local radius of the integral surface.
In the present treatise, however, we are interested in its validity for smooth integral surfaces, but with both the sound source and receiver positions typically located in the proximity of the SSD.
In this scenario it will be shown that the low frequency limit of applicability is mainly dominated by the size of the integral surface (SSD) and the structure of the KA integral's amplitude distribution.

From the integral formulation \eqref{eq:KA} the 3D WFS driving functions could be directly extraced, allowing the synthesis of the target field applying an enclosing SSD surface.
In order to arrive at driving functions for an SSD contour---a scenario termed as 2.5D WFS---the following asymptotic approximation is applied to the 3D KA.

\subsection{The Stationary Phase Approximation}

Consider a Fourier-type contour integral---being a generalization of the boundary integrals given by either the Kirchhoff approximation, or the synthesized field---in the form
\begin{equation}
  I(k) = \oint_C A(s) \te^{\ti k \phi(s)} \td s
  \label{eq:SPA_int}
\end{equation}
with $A$ and $\phi$ being a general, real valued amplitude and phase functions, respectively, and where $s$ is the arc-length parameter along the integration path.
Asymptotically, the integral is dominated by \emph{critical points} in the integral path: discontinuities in any derivative of $A$ or $\phi$, and \emph{stationary positions}, where the phase gradient vanishes.
In the present work, we focus exclusively on the latter contribution, with the resulting asymptotic evaluation termed the stationary phase approximation (SPA).

The SPA relies on the principle that, under high-frequency conditions, rapid oscillations of the integrand lead to mutual cancellation everywhere except in the vicinity of stationary positions.
As a result the integral is dominated by its stationary value.
%The SPA relies on the idea that under high frequency conditions rapid oscillations (due to rapidly varying phase function) may cancel out and the integral is dominated by the stationary positions in the integral path.
In the geometry relevant to the present work the surface integrals separate into products of one-dimensional integrals; hence it suffices to treat the 1-D case, for which the stationary position $s^*$ is found where
\begin{equation}
  \phi'_s(s^{\ast}) \;=\; 0, 
  \qquad
  \phi''_{ss}(s^{\ast})\neq 0
  \label{eq:stationary_point}
\end{equation}
holds, with $\phi'_s$ and $\phi''_{ss}$ signifying the first and second derivatives of the phase function with respect to $s$.

The formal derivation involves the Taylor expansion of both the phase and amplitude functions around $s^*$, followed by term-wise analytical Gaussian integration.
By collecting terms of the same order in $k$, the original integral is rewritten as an asymptotic power series:
\begin{equation}
  I(k) = \sum_{i = 0}^{\infty} \frac{I_i(k)}{  \left(\ti k |\phi_{ss}^{''}(s^*)|\right)^{i}}.
  \label{eq:SPA_series}
\end{equation}
Here, $I_i(k)$ denotes the $i$-th term of the expansion, consisting of powers and derivatives of $A$ and $\phi$ of increasing order.
The SPA of the integral is given by the leading-order, or geometrical-optics term, reading \cite{Bleistein1984, Bleistein2000, Bleistein1986b}
{\fontsize{10pt}{\baselineskip}
\selectfont
\begin{equation}
    I_0(k) \sim % = \sum_{i = 0}^{\infty} \frac{I_i}{\left( k |\Phi^{''}(s^*)| \right)^{i+0.5}} \sim \\
  A(s^*) \te^{\ti k \phi(s^*)} \sqrt{\frac{2\pi}{k |\phi_{ss}^{''}(s^*)|}} \te^{\ti \frac{\pi}{4}\mathrm{sign}(\phi_{ss}^{''}(s^*))},
  \label{eq:SPA0}
\end{equation}}where $\sim$ denotes asymptotic equality as $k \rightarrow \infty$, and $\mathrm{sgn}(\cdot)$ is the signum function.
The terms for $i>0$ represent higher-order asymptotic corrections; for instance, $I_1(k)$ is given by Eq. \eqref{eq:SPA_I1} in Appendix \ref{app:local_SPA}.

%{\fontsize{9pt}{\baselineskip}
%\selectfont
%\begin{multline}
%  I_0(k) \sim % = \sum_{i = 0}^{\infty} \frac{I_i}{\left( k |\Phi^{''}(s^*)| \right)^{i+0.5}} \sim \\
%  \overbrace{A(s^*) \te^{\ti k \Phi(s^*)} \sqrt{\frac{2\pi}{k |\Phi^{''}(s^*)|}} \te^{\ti \frac{\pi}{4}\mathrm{sign}(\Phi^{''}(s^*))}}^{I_0} \cdot \\ \cdot \left(1 + \frac{I_1}{\ti k |\Phi^{''}(s^*)|} +  \frac{I_2}{ ( \ti  k |\Phi^{''}(s^*)|)^2} + ... +  \frac{I_i}{ (\ti  k |\Phi^{''}(s^*)|)^i}  \right)
%  \label{eq:fullSPA}
%\end{multline}}
%As an example, the first non-leading term is given by
%\begin{equation}
%  \footnotesize
%  I_1 = 
%  \frac{1}{2}\frac{A^{''}(s^*)}{A(s^*)} - \frac{ \frac{1}{8}  \Phi^{(4)}(s^*) + \frac{1}{2} \Phi^{'''}(s^*)A'(s^*)/A(s^*)}{\Phi^{''}(s^*)} + \frac{\frac{5}{24} \Phi^{'''}(s^*)^2 }{\Phi^{''}(s^*)^2},
%\end{equation}
%and where each term $I_i$ can be written as a further power series of $\frac{1}{\Phi^{''}(s^*)}$.
%The leading term $I_0$ is the usual form of the stationary phase approximation of the integral, emerging from the Taylor's approximation of the ampltideu and phase up to the leading and the second order respectively.
%Thus, the integral of the function is approximated by its stationary value along with a curvature correction term.

The low frequency validity of the above approximation is of central importance of the present contribution, and will be discussed in the following sections when the formula is applied to integral formulations of sound fields. 

\subsection{The 2.5D Kirchhoff-approximation}
In the following, the SPA is applied to the 3D KA \eqref{eq:KA} to reduce it to a contour integral termed as the 2.5D KA.

The geometry is shown in \ref{Fig:geometry}: the surface of integration $S$ is chosen to be translationally invariant in the vertical direction, allowing to split the 3D KA to a vertical and a horizontal integral ($\int_S ... \td S(\vxo) = \int_{C} \int_{-\infty}^{\infty} ... \td z_0 \td s(x_0,y_0)$), from which the vertical integral is to be evaluated by the SPA.
%In WFS theory the SPA is first applied to the Kirchhoff approximation along the vertical dimension in order to reduce the surface integral to a contour integral along the physically realizable SSD contour. 
The stationary position for the surface integral is found where the gradient of the integrand's phase function vanishes, i.e. where $\vhk^P(\vxo^*) = \vhk^G(\vx-\vxo^*)$ holds (the local propagation direction of the Green's function is aligned with that of the target field along the SSD).
Therefore, for target fields propagating merely along the the plane of investigation with $\hat{k}^P_z = 0$ the vertical stationary position is identically found at $z^* = 0$ as it is illustrated in Figure \ref{Fig:geometry}.

As discussed in detail in \cite{Firtha2025_ieee_a}, the vertical second-order phase derivative present in \eqref{eq:SPA0} is given by $\phi^{P+G''}_{zz} = -\left( \rho^G(\vx-\vxo)^{-1} + \rho^P_v(\vxo)^{-1} \right)$.
Here, $\rho^P_{v/h}$ and $\rho^G$ represent the vertical/horizontal radii of curvature for the target field and the Green's function, respectively \footnote{The principal (horizontal and vertical) radii of the field are defined as the radii of the osculating circles to the wavefront at a given position within the respective principal planes.}.
For a point source, the latter is given identically by $\rho^G(\vx-\vxo) = |\vx-\vxo|$.
%As discussed in details in \cite{Firtha2025_ieee_a} the vertical second phase derivative present in \eqref{eq:SPA0} is given by $\phi^{P+G''}_{zz}  =  -\left( \frac{1}{\rho^G(\vx-\vxo)} + \frac{1}{\rho^P_v(\vxo)} \right)$, with $\rho^P_{v/h}$ and $\rho^G$ being the vertical/horizontal radii of the target field and the Green's function (the latter given identically by $\rho^G(\vx-\vxo) = |\vx-\vxo|$) \footnote{The principal (horizontal/vertical) radii of the field is defined as the radius of largest and smallest osculating spheres to the wavefront at a given position.}.
The above quantity frequently emerges in the following, therefore, for the sake of brevity we define the \emph{vertical/horizontal amplitude correction factor} as
\begin{equation}
  d^{\mathrm{c}}_{v/h}(\vx,\vxo)
    = \frac{\rho^{P}_{v/h}(\vxo) \rho^{G}(\vx- \vxo)}{\rho^{P}_{v/h}(\vxo) + \rho^{G}(\vx- \vxo)}.
\end{equation}

Evaluating \eqref{eq:KA} vertically by the leading order SPA given by Eq.~\eqref{eq:SPA0} results in the contour integral
{\fontsize{8pt}{\baselineskip}
\selectfont
\begin{multline}
  P(\vx,\omega) = \oint_C w(\vxo) \sqrt{\frac{8\pi}{\ti k}} 
  \sqrt{d^{\mathrm{c}}_{v}(\vx,\vxo)}
  P'_{\mathrm{n}}(\vxo, \omega)  G(\vx-\vxo,\omega)\td s(\vxo)
  \label{eq:25D_KA}
\end{multline}}
referred to as the 2.5D Kirchhoff approximation.
Finally, the local plane wave approximation for the target field's gradient given by Eq.~\ref{eq:pw_appr}, a choice justified in the following section, results in the final form
{\fontsize{9pt}{\baselineskip}
\selectfont
\begin{multline}
  P(\vx,\omega) = \oint_C  w(\vxo)  \hkn^P(\vxo) \, \sqrt{8\pi \ti k} \, \sqrt{d^{\mathrm{c}}_{v}(\vx,\vxo)}\cdot
  \\ \cdot 
  P(\vxo, \omega)  G(\vx-\vxo,\omega)\, \td s(\vxo).
  \label{eq:25D_KA_gradient}
\end{multline}}

The above 2.5D integral formulation can be further evaluated asymptotically with decomposing it to an \emph{intended component} and additional \emph{diffractional waves} due to the finite horizontal integral path.
As discussed in details in \cite{Firtha2025_ieee_a} the diffractional components emerge from the discontinuities in the integration path (i.e. from the endpoints of the integral contour and from discontinuities in the window function).
%On the other hand, the intended field---which is the high-frequency asymptotic representation of the target wavefront---can be estimated by the horizontal SPA of \eqref{eq:25D_KA_gradient} around the integral's stationary points $\vxo^*(\vx)$.
%These horizontal stationary position has to be found for each evaluation position $\vx$ (this dependency is not denoted in the following for the sake of brevity).
%Eventually intended field is given by 
On the other hand, the \emph{intended field}---representing the high-frequency asymptotic reconstruction of the target wavefront---is estimated by evaluating the horizontal SPA of \eqref{eq:25D_KA_gradient} around the stationary points $\vxo^*(\vx)$.
%This component corresponds to the contribution of the secondary sources that are in phase with the virtual wavefront at the listener's position.
The horizontal stationary positions must be determined for each evaluation position $\vx$; however, this dependency is omitted in the following for the sake of brevity.
Eventually, the intended field is given by:
{\fontsize{8pt}{\baselineskip}
\selectfont
\begin{multline}
  P(\vx,\omega) \sim 
  4\pi w(\vxo^*) \, \sqrt{d^{\mathrm{c}}_{v}\left(\vx,\vxo^*\right)}   \sqrt{d^{\mathrm{c}}_{h}\left(\vx,\vxo^*\right)}
    P(\vxo^*, \omega) G(\vx-\vxo^*).
   \label{eq:25D_KA_amp_dist} 
\end{multline}}
The above formulation can be shown to coincide with the target field under the validity of the SPA \cite{Firtha2025_ieee_a}.
This fact already shows that the validity of the 2.5D Kirchhoff approximation and the SPA are strongly interconnected: as long the SPA holds the KA inherently holds (asides from the undesired diffractional waves).

From \eqref{eq:25D_KA_amp_dist} the role of the amplitude correction factor is apparent:
In the frequency range where ray theory and the SPA holds, the amplitude distribution of the acoustic field is governed by the geometric spreading of energy, which is inherently defined by the phase function of the field, encoded in the wavefront curvature (as it is reflected by Eq.~(4) \cite{Firtha2025_ieee_a}).
Once the amplitude of the field is known at the stationary position and 3D propagation is assumed, the amplitude is matched to the intended value by the amplitude correction factor.

\subsection{The 2.5D WFS driving functions}

The 2.5D WFS driving functions for an arbitrary SSD contour are derived from the 2.5D KA in \eqref{eq:25D_KA}--\eqref{eq:25D_KA_gradient} by ensuring the integral kernel is independent of the receiver position $\vx$, (as it is apparent in $d^{\mathrm{c}}_{h}\left(\vx,\vxo\right)$).
This is achieved by recognizing that, under the validity of a subsequent horizontal SPA, each SSD element at $\vxo$ dominates the synthesized field at receiver positions for which that element serves as a horizontal stationary position.
This allows the assignment of a unique reference position, $\vxref(\vxo)$, to each secondary source.
As illustrated in Fig.~\ref{Fig:geometry}, $\vxref(\vxo)$ is located along the local wavenumber vector of the target field at an arbitrarily chosen distance from the SSD element.
The locus of all such points defines the reference curve along which amplitude-correct synthesis is achieved.

Under these considerations, the 2.5D WFS driving functions are given by:
{\fontsize{9pt}{\baselineskip}
\selectfont
\begin{subequations}
  \label{eq:25D_WFS}
  \begin{align}
    D(\vxo,\omega) &= w(\vxo) \sqrt{\frac{8\pi} {\ti k}} \sqrt{d^{\mathrm{c}}_{v}\left(\vxref(\vxo),\vxo\right)} P'_{\mathrm{n}}(\vxo, \omega) \label{eq:25D_WFS_exact} \\
    &\approx w(\vxo) \sqrt{8\pi \ti k} \sqrt{ d^{\mathrm{c}}_{v}\left(\vxref(\vxo),\vxo\right) } \hkn^P(\vxo) P(\vxo, \omega). \label{eq:25D_WFS_approx}
  \end{align}
\end{subequations}
}
Similarly to \eqref{eq:25D_KA_amp_dist}, the intended component of the resulting synthesized field is expressed as:
{\fontsize{9pt}{\baselineskip}
\selectfont
\begin{multline}
  \Psynth(\vx,\omega) \sim \\ \sim 4\pi w(\vxo^*) \sqrt{ d^{\mathrm{c}}_{v}\left(\vxref(\vxo^*),\vxo^*\right)}\sqrt{ d^{\mathrm{c}}_{h}\left(\vx,\vxo^*\right)} P(\vxo^*, \omega) G(\vx-\vxo^*).
  \label{eq:WFS_amp_dist} 
\end{multline}}
This formulation recovers the target field exactly at $\vx = \vxref$.
For clarity, while the functional dependency $\vxo^*(\vx)$ is notationally suppressed, evaluating the synthesized field for a specific receiver position $\vx$ requires determining the stationary SSD element $\vxo^*$ a priori, as depicted in Fig.~\ref{Fig:geometry}.


\section{Low frequency limits of WFS}


The primary objective of this contribution is to define a lower cutoff frequency that marks the fundamental operational limit of 2.5D WFS.
While the synthesis relies on multiple high-frequency assumptions, we demonstrate that the low-frequency breakdown of both the 2.5D KA and WFS is dominated by the validity of the stationary phase approximation.
In typical source-receiver geometries, the reconstruction fails specifically when the SPA no longer accurately represents the integral's leading-order contribution.
Identifying the limits of the SPA thus provides a rigorous mathematical criterion for the practical low-frequency cutoff of the synthesis.

%The main goal of the present contribution is to define a lower cutoff frequency above which WFS theory holds.
%The previous section already highlighted that in the geometry under discussion---i.e.~for sources and receivers located typically in the SSD's proximity ---this frequency coincides with the validity of the stationary phase approximation.

The validity of the SPA is a complex problem that would require the careful investigation of the phase and amplitude along the entire integral path~\cite{chassande-mottin_flandrin_1999, 721375}.
A local approximation on the low frequency limit can be given by inspecting the KA's integrand's amplitude and phase changes around the stationary position, as discussed in~\cite{firtha_hahn_schultz_fiala_2025} and briefly outlined in the appendix~\ref{app:local_SPA}.
However, that local approach ignores the finite SSD size, which dominates in WFS.

%In the following a physically motivated approach is introduced, which takes into account finite size SSDs, however, does not investigate the relative rate of amplitude and phase changes in the integration path.
%Therefore, the following approach is mainly applicable for smooth, enclosing SSDs, being a realistic requirement for WFS applications.
%It should be noted that the following discussion both holds in the context of WFS and more generally for the 2.5D Kirchhoff approximation, therefore, they are discussed adjointly.
In the following, a physically motivated approach is introduced that accounts for the finite size of the SSD.
This method focuses on the aperture-limited behavior rather than the relative rates of change between amplitude and phase along the integration path.
Consequently, this approach is primarily applicable to smooth or enclosing SSD contours, which represents a realistic requirement for most WFS applications.\footnote{For non-smooth geometries, such as rectangular SSDs, the low-frequency limit must be evaluated by considering each smooth (linear) segment independently. 
The discontinuities at the corners of such distributions give rise to diffraction artifacts, which are treated separately using the asymptotic framework presented in Part I of this series \cite{Firtha2025_ieee_a}.}
It should be noted that the following discussion holds both for WFS driving functions and, more generally, for the 2.5D KA; therefore, these frameworks are discussed concurrently.

\subsection{Qualitative Analysis}


\begin{figure}[t!]
  \centering
  % LEFT COLUMN
  \begin{minipage}{0.535\columnwidth}
    \begin{overpic}[width=\linewidth]{figs/Fig_2_a_2.png}
      \footnotesize\put(0,2){(a)}
    \end{overpic}
  \end{minipage}
  \hfill
  % RIGHT COLUMN
  \begin{minipage}{0.45\columnwidth}
    \begin{minipage}{1\columnwidth}
      \begin{overpic}[width=\linewidth]{figs/Fig_2_b_2.png}
      \footnotesize\put(0,2){(b)}
    \end{overpic}
  \end{minipage}
    \vspace{0.1cm} % small vertical gap
    
    \begin{minipage}{1\columnwidth}
    \begin{overpic}[width=\linewidth]{figs/Fig_2_c_2.png}
      \footnotesize\put(0,2){(c)}
    \end{overpic}
  \end{minipage}
\end{minipage}
\caption{Evaluation of a virtual point source's field by the 3D and 2.5D KA/WFS. 
  (a) Synthesized pressure field at $f_0 = 1$~kHz using a 2.5D integral contour.% The interior region reconstructs the target field, while the exterior region represents the scattered field; the white-filled area indicates the driving function amplitude distribution $D(\vxo)$ along the SSD. 
  (b) and (c) depict the normalized frequency response $\hat{P}$ at the origin for the 3D and 2.5D cases, respectively. 
  For the simulations, a quasi-continuous SSD was used with a discretization of $\Delta s \approx \Delta z \approx 1$~cm to avoid spatial aliasing within the illustrated range.
  The vertical dimension of the cylinder was truncated at $L_z = \pm 5$~m. 
  %The frequency axis is extended to 1~Hz to clearly illustrate the low-frequency asymptotic trends ($\sim 1/jk$ and $\sim 1/\sqrt{jk}$ slopes) predicted by the theory. 
  The value $\hat{P}$ denotes the synthesized pressure normalized by the target field's nominal amplitude at the receiver position.}
 \label{fig:Kirchhoff_appr}
\end{figure}
For a qualitative analysis of the underlying principles, we first investigate 3D and 2.5D WFS reproducing a virtual point source located on the synthesis plane.
Fig.~\ref{fig:Kirchhoff_appr}(a) illustrates the synthesized pressure field, which serves as an approximation of the target field within the interior of the SSD and as an approximation of the scattered field for the equivalent scattering problem in the exterior domain.
Figures~\ref{fig:Kirchhoff_appr}(b) and (c) depict the 3D and 2.5D frequency responses evaluated at the center of the SSD, normalized by the nominal amplitude of the target field.
In this example, the SSD is a cylindrical surface for the 3D case and a circular contour for the 2.5D case.

To investigate the structure of the synthesized field, the gradient decomposition of the target field from \eqref{eq:gradient} is applied to both the 3D and 2.5D WFS integrals.
This allows the field to be partitioned into near-field and far-field contributions:
{\fontsize{8pt}{\baselineskip}
\selectfont
\begin{multline}
 % \Psynth(\vx,\omega)  = \oint_C w(\vxo) \, \sqrt{\frac{8\pi} {\ti k}} \, \sqrt{ d^{\mathrm{c}}_{v}\left(\vxref(\vxo),\vxo\right)} \left( \overline{\kappa}^P(\vxo) + \ti k\right) \cdot \\ \cdot 
 % \hkn^P(\vxo) P(\vxo, \omega) G(\vx-\vxo,\omega) \td s(\vxo) 
 % = I_{\mathrm{near}}(\vx,\omega) + I_{\mathrm{far}}(\vx,\omega) 
 \Psynth(\vx,\omega)  = \oint_{C/S} D(\vxo) G(\vx-\vxo,\omega) \td s/S(\vxo)  = \\
 = I_{\mathrm{near}}(\vx,\omega) + I_{\mathrm{far}}(\vx,\omega).
\end{multline}}
For the 2.5D case, utilizing the driving functions defined in \eqref{eq:25D_WFS_exact}, the components of the synthesized field are expressed as:
\begin{subequations}
\label{eq:I_near_far}
{\fontsize{9pt}{\baselineskip} \selectfont
\begin{align}
  I_{\mathrm{near}}(\vx,\omega) &= \sqrt{\frac{8\pi}{\ti k}} \oint_C \overline{\kappa}^P(\vxo) w(\vxo) \hkn^P(\vxo) \sqrt{d^{\mathrm{c}}_{v}} P \cdot G \td s(\vxo), \\
  I_{\mathrm{far}}(\vx,\omega) &= \sqrt{8\pi \ti k} \oint_C w(\vxo) \hkn^P(\vxo) \sqrt{d^{\mathrm{c}}_{v}} P \cdot G \td s(\vxo) = \nonumber \\
  &= \sqrt{\ti k} \oint_C A^{D \cdot G}(\vx, \vxo) \te^{\ti k \phi^{P\cdot G}(\vxo)} \td s(\vxo), \label{eq:Ifar}
\end{align}}
\end{subequations}
where the arguments of $d^{\mathrm{c}}_{v}$, $P$, and $G$ follow the definitions in \eqref{eq:25D_WFS}.
The magnitude and phase of the far-field integrand are determined by the combination of the target field and the Green's function as follows:
\begin{subequations}
\label{eq:amp_phase_def}
{\fontsize{8pt}{\baselineskip} \selectfont
\begin{align}
  \small
  A^{D\cdot G}(\vx,\vxo) &= \sqrt{8\pi} w(\vxo) \hkn^P(\vxo) \sqrt{d^{\mathrm{c}}_{v}(\vxref, \vxo)} A^P(\vxo) A^G(\vx - \vxo), \label{eq:wfs_amp} \\
  \phi^{P\cdot G}(\vx,\vxo) &= \phi^P(\vxo) + \phi^G(\vx-\vxo). \label{eq:wfs_phase}
\end{align}}
\end{subequations}
%Note that the near-field and far-field components are subject to constant phase shifts of $-\pi/4$ and $+\pi/4$, respectively, relative to the combined phase $\phi^{P\cdot G}$. 
A similar decomposition can be derived for the 3D case by omitting the 2.5D correction factor $\sqrt{8\pi/\ti k} \sqrt{d^{\mathrm{c}}_{v}}$.

The amplitude distribution of the integrand given by \eqref{eq:wfs_amp} is illustrated in Fig.~\ref{fig:Kirchhoff_appr} (a) over the SSD with white area.
%It is important to realize that even in case of an infinite planar/linear integration path the amplitude distribution decays rapidly towards the lateral integration positions, therefore, the amplitude distribution acts as a windowing function with an effective length defined in the following section.% \footnote{Similar effective aperture sizes can be defined for the 3D case as well. For the 3D case, however, the problem is more complex as in the present separable geometry a vertical and horizontal effective length can be defined and the behaviour of the integrals change at two cutoff frequencies, depending on these lengths. 
%It is important to recognize that even in the case of an infinite integration path—such as the vertical integral in the present cylindrical geometry or an infinite linear contour—the integrand's amplitude decays rapidly toward lateral positions.
Notably, the integrand amplitude decays rapidly toward lateral positions—a property not limited to enclosing geometries, but universal to Kirchhoff integrals (including infinite lines). 
This occurs because the gradient projection $\hkn^P(\vxo)$ vanishes as the integration point moves laterally.
%This natural decay acts as an inherent windowing function, defining an effective aperture length that restricts the region contributing constructively to the resulting field.
This decay defines an \emph{effective aperture length}, which restricts the region of constructive contribution.
%The relative contributions of $I_{\mathrm{near}}$ and $I_{\mathrm{far}}$ are determined by the total phase variation \eqref{eq:wfs_phase} within this effective window.
Consequently, the relative dominance of $I_{\mathrm{near}}$ and $I_{\mathrm{far}}$ depends on the total phase variation of $\phi^{P\cdot G}$ across this aperture.
%This aspect is, however, out of the scope of the present discussion.}.
%The contribution of $I_{\mathrm{near}}$ and $I_{\mathrm{far}}$ depend on how the phase function $\phi^P(\vxo) + \phi^G(\vx-\vxo)$ changes within this effective window length.
%The synthesis behavior can be categorized into two distinct frequency regimes based on the phase variation across the effective aperture $L_{\mathrm{eff}}$:
\begin{itemize}
    \item% \textbf{Low-Frequency Regime (Breakdown of SPA):} 
    At low frequencies, the phase $\phi^{P\cdot G}$ varies minimally across the effective aperture length, and the integrands of \eqref{eq:I_near_far} are approximately constant over the aperture.
    In the 3D case, as seen in Fig.~\ref{fig:Kirchhoff_appr}(b), $I_{\mathrm{near}}$ remains constant while $I_{\mathrm{far}}$ is proportional to $jk$.
    In the 2.5D case, these are scaled by the $\sqrt{1/jk}$ correction factor, resulting in a $\sqrt{1/jk}$ boost for the near-field and a $\sqrt{jk}$ slope for the far-field component in Fig.~\ref{fig:Kirchhoff_appr}(c).
    The constant near-field level error in 3D reveals that the KA is inherently unable to reconstruct the target field's evanescent or reactive near-field components.

    \item%\textbf{High-Frequency Regime (Validity of SPA):} 
    As frequency increases, rapid phase oscillations lead to mutual cancellation outside the stationary region, allowing the SPA to hold.
    The integral's value is then determined by the stationary value scaled by the SPA's characteristic frequency decay: $1/jk$ for 3D surface integrals and $1/\sqrt{jk}$ for 2.5D contour integrals.
    In both cases, this asymptotic decay exactly compensates the frequency dependency of the far-field driving function, yielding the desired flat frequency response.
    Simultaneously, the near-field term $I_{\mathrm{near}}$ vanishes at a rate of $1/jk$ as the stationary phase condition is fulfilled.
\end{itemize}

The transition between these regimes—where the far-field response flattens—defines the operational limit of the theory.
The following analysis aims to quantify the phase increment over the effective apperture length required to reach this asymptotic state, thereby establishing the practical low-frequency cutoff for 2.5D WFS.


\subsection{Relation of the SPA validity and Fresnel zone theory}
A preliminary estimate for the cutoff frequency can be derived by inspecting the asymptotic behavior of the far-field component $I_{\mathrm{far}}$.
In the low-frequency limit, where the phase $\phi^{P\cdot G}$ is approximately constant across the SSD, the magnitude of the synthesized field is governed by the integral of the amplitude distribution:
\begin{equation}
    |I_{\mathrm{far}}(\vx, \omega \to 0)| \approx \sqrt{k} \oint_C A^{D\cdot G}(\vx, \vxo) \td s(\vxo).
    \label{eq:low_freq_asymptote}
\end{equation}
This represents a $+10$~dB/decade slope ($\sim \sqrt{k}$) as observed in Fig.~\ref{fig:Kirchhoff_appr}(c).
As the frequency increases, the field eventually reaches its stable asymptotic value defined by the high-frequency SPA reconstruction in \eqref{eq:WFS_amp_dist}.
The magnitude of this high-frequency asymptote is given by:
\begin{equation} 
    |I_{\mathrm{far}}(\vx, \omega \gg \omega_c)| \approx 4\pi w(\vxo^*) \sqrt{d^{\mathrm{c}}_{v} d^{\mathrm{c}}_{h}} A^P(\vxo^*) A^G(\vx-\vxo^*),
\end{equation} 
assuming a stationary point $\vxo^*$ within the active SSD.
The cutoff wavenumber $k_c$ is defined as the intersection of these two asymptotic trends. 
Equating the expressions yields the geometric estimate:
\begin{equation}
    k_c(\vx) = \left( \frac{ 4\pi w(\vxo^*) \sqrt{d^{\mathrm{c}}_{v} d^{\mathrm{c}}_{h}} A^P(\vxo^*) A^G(\vx-\vxo^*)}{\oint_C A^{D\cdot G}(\vx, \vxo) \td s(\vxo)} \right)^2.
    \label{eq:omega_c}
\end{equation}

While this formulation allows for a quick estimation of the 2.5D WFS limit using purely geometrical considerations, it provides high accuracy only for on-axis listener positions.
To arrive at a more general result, we must investigate the underlying physical mechanism: the total phase increment accumulated across the SSD at this cutoff frequency.
\begin{figure} 
  \begin{center}
    \begin{overpic}[width = 0.55\columnwidth  ]{figs/effective_window_simple_geom.png}
    \footnotesize
    \put(80.5,39){$\vxo^*(\vx)$}
    \put(38,39){$\vxref(\vxo^*)$}
    \put(22,39){$\vx$}
    \put(95,30){$\vk^{\mathrm{PW}}$}  
    \put(75,57){\textcolor[RGB]{162,50,52}{$\Leff$}}  
    \scriptsize \put(68,28){\rotatebox{90}{ \textcolor[RGB]{107,107,107}{ $A^{D\cdot G}(s)$} } }  
    %\put(29,37){$\vx = \vxref(\vxo^*)$}
	\end{overpic}
\end{center}
\caption{Symmetrical geometry for estimating the phase increment towards the SSD endpoints at the cutoff frequency in case of an incident plane wave with the wave vector $\vk^{\mathrm{PW}}$.
The amplitude distribution $A^{D\cdot P}(\vx,\vxo)$ is denoted by white filled area, while the symmetrical equivalent window is denoted by red fill.}
	\label{Fig:Leff_simplified_geometry}
\end{figure}

As an initial geometry, consider the symmetric case depicted in Fig.~\ref{Fig:Leff_simplified_geometry}, corresponding to the example in Fig.~\ref{fig:Kirchhoff_appr}. 
The SSD is chosen to be a circular (albeit potentially arbitrary symmetric) one.
Assuming both the virtual source and the receiver are sufficiently far from the SSD, the amplitude terms $A^P$ and $A^G$ vary minimally across the contributing region, justifying a leading-order approximation.
However, we explicitly account for the variation of the tapering window $w(\vxo)$ and the projected gradient $\hkn^P(\vxo)$, both of which vanish toward the SSD lateral directions.
Under these assumptions, the low-frequency asymptotic behavior of the far-field component is approximated as:
\begin{multline} 
|I_{\mathrm{far}}(\vx, \omega \to 0)| \approx \sqrt{k} \cdot \sqrt{8\pi} \, \sqrt{d^{\mathrm{c}}_{v}} \, A^P(\vxo^*) \, A^G(\vx - \vxo^*) \cdot \\ 
\cdot \underbrace{\oint_C w(\vxo) \hkn^P(\vxo) \td s(\vxo)}_{2\Leff },
\label{eq:Ifar_appr}
\end{multline}
where $\Leff$ represents an equivalent, one-sided \emph{effective window/aperture length} (measured from the central/stationary position).
For an incident plane wave with the receiver at the SSD center, \eqref{eq:Ifar_appr} holds with equality.
Substituting this approximation into \eqref{eq:omega_c}, the cutoff wavenumber simplifies to:
\begin{equation} 
k_c(\vx) \approx \frac{\pi}{2} \frac{ d^{\mathrm{c}}_{h}\left(\vx,\vxo^*\right) }{ \Leff^2 }.
\label{eq:omega_c_appr}
\end{equation}

% By substituting the above approximation to \eqref{eq:omega_c} cutoff wavenumber then can be approximated in the present setup by
% {\fontsize{10pt}{\baselineskip}
% \selectfont
% \begin{equation} 
% k_c(\vx) \approx
%  \frac{\pi}{2} 
%  \frac{ d^{\mathrm{c}}_{h}\left(\vx,\vxo^*\right) }{ 
%    \Leff^2
%  }.
% \label{eq:omega_c_appr}
% \end{equation}}

The validity of the SPA can be linked to the phase increment $\Delta \phi^{P \cdot G}$ from the stationary point $\vxo^*$ to the effective aperture boundary.
As derived in Appendix~\ref{app:phase_increment}, the second-order Taylor expansion of the phase variation over an arc length $\Delta s$ is:
\begin{equation} 
 \Delta \phi^{P \cdot G}(\Delta s) = - \frac{\Delta s^2}{2} \frac{\hkn^P(\vxo^*)^2}{d^{\mathrm{c}}_{h}(\vx, \vxo^*)}.
 \label{eq:phase_inc_main}
\end{equation}
Evaluating this phase change at the cutoff wavenumber $k_c$ for the effective length $\Delta s = \Leff$, and assuming a frontal incidence ($\hkn^P(\vxo^*) = 1$), we obtain:
\begin{equation} 
 k_c \Delta \phi^{P \cdot G}(\Leff) = -\frac{\pi}{4}.
\end{equation}
This result is fundamental: it demonstrates that in frontal directions, the SPA—and thus accurate WFS reconstruction—commences when the phase increment from the center to the edge of the effective aperture reaches $\pi/4$. 
In terms of total aperture extent, this implies the effective window $2\Leff$ must span at least one-quarter of a half-wavelength.

This observation aligns with Fresnel zone theory, generalized here for a 1D aperture with an arbitrary complex incident distribution.
Eq \eqref{eq:omega_c_appr} can be rearranged to
\begin{equation} 
 N_F = \frac{\Leff^2}{\lambda_c d^{\mathrm{c}}_{h}(\vx, \vxo^*)} = 0.25,
\end{equation}
where $\lambda_c = 2\pi/k_c$, which is clearly the Fresnel number, generalized for a 1D aperture with an arbitrary complex incident distribution.
In this context, the effective length $\Leff$ acts as the aperture radius, while $d^{\mathrm{c}}_{h}$ serves as an effective propagation distance.
For the special case of an incident plane-wave ($\rho^P_{v/h} \to \infty$), $d^{\mathrm{c}}_{h}$ simplifies to the receiver-to-SSD distance, recovering the classical Fresnel number definition.

Overall, it is established that a Fresnel number of $N_F \geq 0.25$ is required for the SPA to hold.
The connection between SPA validity and Fresnel theory has been already reported for 2.5D Kirchhoff migration \cite{Hertweck2003}, stating that integration over the first Fresnel zone ($N_F = 1$) provides a sufficient approximation of the total integral (requiring a phase change of at least $\pi$ from the center to the aperture end). 
However, there the effective aperture length (i.e. the apparent shortening of the aperture) is not taken into consideration, which explains the apparently looser condition, introduced here.

%This result highlights a crucial characteristic of WFS: while based on a high-frequency asymptotic derivation, the method effectively operates as a near-field approximation.
%As the distance $d^{\mathrm{c}}_{h}$ from the SSD increases, the generalized Fresnel number $N_F$ decreases for a fixed frequency.
%Consequently, if the receiver is too far from the secondary source distribution, the condition $N_F \geq 0.25$ is violated, and the synthesized field collapses into a directive spherical wave.
%Maintaining the intended wavefront shape thus requires the listener to remain within a specific proximity to the SSD, where the effective aperture is large enough to support the stationary phase reconstruction.

\subsection{General low frequency validity condition for 2.5D WFS}
In the following, we generalize the previous results to establish low-frequency validity conditions for 2.5D WFS applicable to arbitrary source models and receiver positions.

\begin{figure} 
  \begin{center}
    \begin{overpic}[width = 0.8\columnwidth]{figs/effective_window_geom.png}
	\footnotesize
    \put(48,11.5){$\vxo^*(\vx)$}
    \put(25,27){$\vxref(\vxo^*)$}
    \put(28,39){$\vx$}
    \put(44,4){$\vx_1$}
    \put(81,17){$\vx_2$}
    \put(52,3.5){\rotatebox{20}{ \textcolor[RGB]{162,50,52}{$\hat{L}_{\mathrm{eff},1}$}}  }
    \put(67,9){\rotatebox{21}{ \textcolor[RGB]{162,50,52}{$\hat{L}_{\mathrm{eff},2}$}}  }
    \scriptsize \put(60,18){\rotatebox{15}{ \textcolor[RGB]{107,107,107}{ $A^{D\cdot P}(s)$} } }  
  \put(35,50){$\mathfrak{I}$}
    \put(20,16){$\mathfrak{T}_1$}
  \put(85,40){$\mathfrak{T}_2$}
  \put(30,8.5){$\mathfrak{S}_1$}
  \put(90,29){$\mathfrak{S}_2$}
    \put(56,57){$C: s(\vxo)$}
	\end{overpic}
\end{center}
\caption{General geometry for estimating the low frequency cutoff for 2.5D WFS.
The amplitude distribution $A^{D\cdot P}(\vx,\vxo)$ is denoted by white filled area, while the asymmetrical equivalent window is denoted by red fill.
$\mathfrak{I}$ denotes the illuminated region, $\mathfrak{S}_i$ the shadow and $\mathfrak{T}_i$ the tapering regions \cite{Firtha2025_ieee_a}.}
	\label{Fig:Leff_geometry}
\end{figure}

The role of the effective window length emerged naturally in \eqref{eq:Ifar_appr} by considering variations in the tapering window and the projected wavenumber vector.
In a more general framework, the amplitude variations of the target field and the Green's function also contribute to this windowing effect as they decay toward the SSD endpoints.
Furthermore, non-central stationary positions must be accounted for to accommodate arbitrary SSD shapes and lateral receiver displacements.
We define the effective window lengths, $\hat{L}_{\mathrm{eff},1/2}(\vx)$, measured from the stationary position toward both ends of the SSD, denoted by $\vx_{1/2}$ in Fig.~\ref{Fig:Leff_geometry}.
These lengths are derived from the amplitude distribution of the integrand \eqref{eq:wfs_amp}, normalized by its value at the stationary point:
 {\fontsize{10pt}{\baselineskip}
\selectfont
\begin{equation}
\hat{L}_{\mathrm{eff},1/2}(\vx) = \int_{\vxo^*(\vx)}^{\vx_{1/2}} \frac{  A^{D\cdot P}(\vx,\vxo) }{A^{D\cdot P}(\vx,\vxo^*(\vx))} \td s(\vxo).
\label{eq:Leff}
\end{equation}}
%where $A^{D\cdot P}$ is the amplitude of the integrand describing the synthesized field, i.e. the product of the driving function's and the Green's function's amplitude
% {\fontsize{8pt}{\baselineskip}
%\selectfont 
%\begin{equation}
%A^{D\cdot P}(\vx,\vxo) = w(\vxo)  \hkn^P(\vxo) \,\sqrt{d^{\mathrm{c}}_{v}\left(\vxref(\vxo),\vxo\right)}  A^P(\vxo)  A^G(\vx-\vxo).
%\end{equation}}
Note that in the previous symmetric case, explicit normalization was unnecessary as the tapering window and the projected wavenumber vector are inherently normalized quantities.
For the SPA to hold, we require the phase increment to reach at least $\pi / 4$ across both effective window lengths.
This criterion defines two cutoff wavenumbers:
\begin{equation}
  k_{c,1/2}(\vx) \cdot \, \Delta \phi(\hat{L}_{\mathrm{eff},1/2}(\vx)) = -\frac{\pi}{4}.
\end{equation}
By substituting the phase increment from \eqref{eq:phase_inc}, the cutoff wavenumbers are given by:
\begin{equation}
  \small
  k_{c,1/2}(\vx) = \frac{\pi}{2} \frac{ d^{\mathrm{c}}_{h}(\vx,\vxo^*)}{
   \hat{L}_{\mathrm{eff},1/2}(\vx) ^2 \hat{k}_{\mathrm{n}}(\vxo^*)^2  }
  \label{eq:low_freq_limit}
\end{equation}
with the corresponding cutoff frequencies $f_{c, 1/2}(\vx) = \frac{k_c(\vx) c}{2 \pi}$.
The cutoff frequency $f_c$ represents the lower limit of the WFS reproduction band; above this frequency, the desired flat magnitude response is asymptotically achieved.
While Fig.~\ref{fig:Kirchhoff_appr} shows this limit at very low frequencies for distant sources, its practical relevance increases significantly in constrained geometries.
Equation \eqref{eq:low_freq_limit} reveals a dual dependency on the source and receiver distances.
As the distance $d^{\mathrm{c}}_{h}$ increases, the cutoff frequency rises, eventually causing the synthesized field to collapse into a directive spherical wave as the Fresnel condition $N_F \geq 0.25$ is violated.
Maintaining the intended wavefront shape thus requires the listener to remain within a specific proximity to the SSD, where the effective aperture is large enough to support the stationary phase reconstruction.
Conversely, if either the source or the receiver is in the immediate proximity of the SSD, the amplitude distribution becomes highly peaked.
This significantly shortens the effective aperture length $\hat{L}_{\mathrm{eff}}$, which again leads to an increase in the cutoff frequency.
Consequently, an optimal frequency band for synthesis is found at moderate distances where the effective aperture is sufficiently large but the propagation distance remains within the near-field reconstruction zone.


%and $f_{c, 1/2}(\vx) = \frac{k_c(\vx) \, c}{2 \pi}$, being the main result of the present contribution.
%The above cutoff frequency is obviously a generalization of \eqref{eq:omega_c_appr}, extended towards arbitrary integrand amplitude and lateral source/receiver positions.
%The expression states that the frequency band of synthesis is maximal for frontal positions ($\hat{k}_{\mathrm{n}}(\vxo^*) = 1$), and for sources and receivers far from the SSD. 
%However it may dramatically decrease once either the source or the receiver gets to the SSD proximity.
%Furthermore, the lower cutoff frequency scales inversely proportionally with the second power of the SSD's effective length.
%This effective length is dominated by the SSD's physical dimensions, but is also affected by the source and receiver positions (again, sources or receivers in the proximity of the SSD significantly shorten the effective length) and can be altered by applying tapering or unsuitable referencing scheme.

%As a brief summary, in order to calculate the low frequency limit of 2.5D WFS in a practical scenario one has to define the stationary position for a given receiver position by simple geometric means.
%Having found the stationary position the equivalent/effective window length has to be calculated on both sides of the stationary point, requiring the numerical integration of the driving function (multiplied by the Green's function).
%From the effective window length the lower cutoff frequencies can be calculated analytically by \eqref{eq:low_freq_limit}, from which $\max\{f_{c, 1},f_{c, 2} \}$ gives the overall cutoff frequency (defined from the shorter window length).
In summary, the low-frequency limit of 2.5D WFS can be calculated in three steps.
First, determine the stationary position $\vxo^*$ for the given receiver through geometric considerations.
Second, evaluate the effective half-lengths $\hat{L}_{\mathrm{eff},1/2}$ by numerically integrating the amplitude of the integrand on both sides of $\vxo^*$ using \eqref{eq:Leff}.
Finally, the overall cutoff frequency is determined by $f_c = \max\{f_{c, 1}, f_{c, 2} \}$, where each $f_{c,i}$ is calculated analytically via \eqref{eq:low_freq_limit}.
This maximum value ensures that the reconstruction condition is satisfied across the entire active aperture, typically dictated by the more restrictive (shorter) half-window.

\section{Application examples}

\subsection{Cutoff frequency for simple WFS geometries}
Next, the application of the previous results is investigated for practical 2.5D WFS scenarios: in case of the synthesis of point sources and a plane waves using linear and circular SSDs.

\paragraph{Cutoff frequency for a virtual point source}
For a virtual point source located in the synthesis plane at $\vxs = \posvec{3}{x_s}{y_s}{0}$, the principal radii and the local wavenumber vector specialize to $\rho^P_{\mathrm{v/h}}(\vx) = |\vx-\vxs|$ and $\vk^P(\vx) = (\vx-\vxs)/|\vx-\vxs|$.
The amplitude of the integrand in \eqref{eq:Ifar} is then given by:
 {\fontsize{7.5pt}{\baselineskip}
\selectfont
\begin{multline}
A^{D\cdot G}(\vx,\vxo) = w(\vxo) \, \sqrt{\frac{ |\vxref-\vxo| }{F\,|\vxref-\vxo| + |\vxo - \vxs|}} \frac{(\vxo-\vxs) \cdot \mathbf{n}(\vxo) }{|\vxo-\vxs|^{5/2}},
\end{multline}
 }
where the flag $F$ accounts for virtual sources located outside ($F=1$) or inside ($F=-1$) the SSD, the latter being termed \emph{focused virtual sources} \cite{Firtha2025_ieee_a}.
Once the reference curve is defined via $\vxref(\vxo)$, the effective lengths can be calculated from the amplitude distribution according to \eqref{eq:Leff}.
Finally, the low-frequency cutoff wavenumber is calculated as:
\begin{equation}
  \small
  k_{c,1/2}^{\mathrm{PS}}(\vx) = \frac{\pi}{2} 
  \frac{|\vx-\vxo^*|\,|\vxo^* - \vxs| }{F |\vx-\vxo^*| + |\vxo^* - \vxs|}
  \frac{1}{ \hat{L}_{\mathrm{eff},1/2}(\vx) ^2 \hat{k}_{\mathrm{n}}(\vxo^*)^2 },
  \label{eq:low_freq_limit_PS}
\end{equation}
where for any receiver position $\vx$, the corresponding stationary SSD element $\vxo^*(\vx)$ lies at the intersection of the SSD and the line connecting the virtual source and the receiver (see e.g., Fig.~\ref{Fig:Psynth_circular}).


%\begin{figure} 
%  \begin{center}
%    \begin{overpic}[width = 1\columnwidth]{figs/Psynth_linear_ps.png}
%            \scriptsize \put(0,10){(a)}
%            \scriptsize \put(58,0){(b)}
%	\end{overpic}
%\begin{minipage}[width = 0.5\columwidth]
%    \begin{overpic}[width = 1\columnwidth]{figs/Psynth_linear_ps_c.png}
%     \scriptsize \put(0,10){(a)}
%  	\end{overpic}
%\end{minipage}
%\begin{minipage}[width = 0.5\columwidth]
%    \begin{overpic}[width = 1\columnwidth]{figs/Psynth_linear_ps_a.png}
%     \scriptsize \put(0,10){(a)}
%  	\end{overpic}
%    \begin{overpic}[width = 1\columnwidth]{figs/Psynth_linear_ps_b.png}
%     \scriptsize \put(0,10){(a)}
%  	\end{overpic}
%\end{minipage}
%\end{center}
%\caption{Synthesis of a point source by using a finite linear SSD segment with the total length of 3~m.
%Figure (a) depicts the synthesized field along with the investigation setup 
%Figure (b) shows the frequency response in the frontal ($\vx_{\mathrm{r}_1} = \posvec{2}{0}{1}~\mathrm{m}$) and an examplery lateral ($\vx_{\mathrm{r}_1} = \posvec{2}{1.5}{1}~\mathrm{m}$) receiver position, both chosen to lie on the reference line in the present example, with highlighting the estimated lower cutoff frequencies.}%%
%	\label{Fig:Psynth_linear}
%\end{figure}

% preamble:
% \usepackage{graphicx}
% \usepackage[percent]{overpic}
% Preamble:
% \usepackage{graphicx}
% \usepackage[percent]{overpic} % for \put coords in %
% (no extra packages needed for \dimexpr)

\begin{figure}
  \centering
  % One horizontal box that holds both columns
  \makebox[\linewidth][c]{%
    % Left column
    \begin{minipage}[t]{\dimexpr.5\linewidth-.5\columnsep\relax}\vspace*{0pt}
      \begin{overpic}[width=\linewidth]{figs/Psynth_linear_ps_a.png}
        \scriptsize \put(0,10){(a)}
      \end{overpic}

      \vspace{0.75ex}

      \begin{overpic}[width=\linewidth]{figs/Psynth_linear2_ps_a.png}
        \scriptsize \put(0,10){(d)}
      \end{overpic}
    \end{minipage}%
    \hspace{\columnsep}% exact column gap
    % Right column
    %
    \begin{minipage}[t]{\dimexpr.5\linewidth-.5\columnsep\relax}\vspace*{0pt}
      \vspace{1mm}
      \begin{overpic}[width=\linewidth]{figs/Psynth_linear_ps_b.png}
        \scriptsize \put(0,0){(b)}
      \end{overpic}

      \vspace{0.75ex}
      \vspace{0mm}

      \begin{overpic}[width=\linewidth]{figs/Psynth_linear_ps_c.png}
        \scriptsize \put(0,0){(c)}
      \end{overpic}

      \vspace{0.75ex}

      \vspace{2.5mm}
      \begin{overpic}[width=\linewidth]{figs/Psynth_linear2_ps_b.png}
        \scriptsize \put(0,0){(e)}
      \end{overpic}
    \end{minipage}%
  }% end makebox

  \caption{Synthesis of a point source using a finite linear, quasi-continuous SSD segment of length $L=4$\,m with $\Delta x_0 = 1~\mathrm{cm}$.
 % The normalized response $\hat{P}$ is shown relative to the target field amplitude.
  Fig. (a-c) shows the synthesis of a virtual source at $\vx_{\mathrm{s}}=\posvec{2}{0}{-1}\,\mathrm{m}$ with (b-c) illustrating the normalized frequency response at in the central ($\vx_{\mathrm{r}_1}=\posvec{2}{0}{1.5}\,\mathrm{m}$) and an exemplary off-center ($\vx_{\mathrm{r}_2}=\posvec{2}{1.5}{1}\,\mathrm{m}$) receiver position on the reference line.
  Fig. (d-e) shows the synthesis of a virtual source at $\vx_{\mathrm{s}}=\posvec{2}{2}{-1}\,\mathrm{m}$, with the frequency response measured at $\vx_{\mathrm{r}_1}=\posvec{2}{0}{1.5}\,\mathrm{m}$.
  Estimated cutoff frequencies $f_{c,1/2}$ are highlighted by dashed lines.}
  \label{Fig:Psynth_linear}
\end{figure}

\paragraph{Cutoff frequency for a virtual plane wave}
For the special case of a virtual plane wave propagating in the direction $\vk^{\mathrm{PW}}$, the principal radii $\rho^P_{\mathrm{v/h}}(\vx) \to \infty$, and the integrand amplitude specializes to:
\begin{equation}
A^{D\cdot G}(\vx,\vxo) = w(\vxo) \left( \vk^{\mathrm{PW}} \cdot \mathbf{n}(\vxo) \right) \frac{\sqrt{|\vxref-\vxo|}}{|\vx-\vxo|}.
\end{equation}
The stationary SSD element $\vxo^*(\vx)$ is found at the intersection of the SSD and the line passing through $\vx$ with the direction $\vk^{\mathrm{PW}}$ (see Fig.~\ref{Fig:Psynth_circular}).
This allows the calculation of the cutoff wavenumber as:
\begin{equation}
  \small
  k_{c,1/2}^{\mathrm{PW}}(\vx) = \frac{\pi}{2} \frac{|\vx-\vxo^*|}{ \hat{L}_{\mathrm{eff},1/2}(\vx) ^2 \hkn^P(\vxo^*)^2 }.
  \label{eq:low_freq_limit_PW}
\end{equation}

\paragraph{Application with linear and circular SSD}
Figure \ref{Fig:Psynth_linear} presents the synthesis of a point source using a linear SSD along the $y = 0$ axis. 
The reference curve is chosen as a line parallel to the SSD at $y = \yref$, corresponding to the traditional WFS geometry for which the theory was first formulated \cite{Berkhout1988}.
In this configuration, the distance ratio simplifies via triangle similarities to $\frac{|\vxref-\vxo| }{|\vxref-\vxo| + |\vxo - \vxs|} = \frac{\yref}{\yref - \ys}$ for the virtual point source case.

Figure \ref{Fig:Psynth_circular} illustrates the synthesis of a plane wave (a), an unfocused point source (b), and a focused point source (c) using a circular SSD of radius $\Rssd = 2~\mathrm{m}$ centered at the origin.
As a simple yet widely used referencing scheme, the reference distance is fixed to $|\vxref(\vxo)| = \Rssd$ for all cases, resulting in a reference curve that varies with the virtual source model while always passing through the center of the array \cite{Firtha2016, Firtha2025_ieee_a}.
  
\begin{figure} 
  \begin{center}
    \begin{overpic}[width = 1\columnwidth ]{figs/Psynth_circular_pw.png}
      \scriptsize \put(0,0){(a)}
      \scriptsize \put(45,0){(b)}
	\end{overpic}
 \\ \vspace{5mm}
    \begin{overpic}[width = 1\columnwidth ]{figs/Psynth_circular_ps.png}
      \scriptsize \put(0,0){(c)}
      \scriptsize \put(45,0){(d)}
	\end{overpic}
 \\ \vspace{5mm}
    \begin{overpic}[width = 1\columnwidth ]{figs/Psynth_circular_fps.png}
      \scriptsize \put(0,0){(e)}
      \scriptsize \put(45,0){(f)}
	\end{overpic}
\end{center}
\caption{
  %Synthesis of a plane wave (a-b), an unfocused point source at $\vxs = \posvec{2}{3}{0}~\mathrm{m}$ (c-d) and a focused point source at at $\vxs = \posvec{2}{1}{0}~\mathrm{m}$ (e-f) by using a circular SSD with $\Rssd = 2~\mathrm{m}$.
  %Figure (a,c,e) depicts the synthesized field along with the investigation setup 
  %Figure (b,d,f) shows the frequency response in the frontal ($\vx_{\mathrm{r}_1} = \posvec{2}{0}{0}~\mathrm{m}$) and an examplery lateral ($\vx_{\mathrm{r}_1} = \posvec{2}{0}{1.5}~\mathrm{m}$) receiver position, both chosen to lie on the reference line in the present example, with highlighting the estimated lower cutoff frequencies.
Synthesis of (a--b) a plane wave, (c--d) a point source at $\vxs = [3, 0, 0]^\top$\,m, and (e--f) a focused point source at $\vxs = [1, 0, 0]^\top$\,m using a circular SSD ($R_{\mathrm{SSD}} = 2$\,m). 
Left panels (a, c, e) depict the synthesized pressure fields and the investigation setup; the white fill along the SSD represents the integrand amplitude for the off-center case, while red highlights denote the calculated effective lengths $\hat{L}_{\mathrm{eff},1/2}$. 
Right panels (b, d, f) show the normalized frequency response $\hat{P}$ at an on-center ($\vx_{\mathrm{r}_1} = [0, 0, 0]^\top$\,m) and an exemplary off-center ($\vx_{\mathrm{r}_2} = [0, 1.5, 0]^\top$\,m) receiver position. 
The dashed lines indicate the estimated lower cutoff frequencies $f_{c,1/2}$.%, marking the transition to the high-frequency WFS plateau. 
}
	\label{Fig:Psynth_circular}
\end{figure}

For both geometries, the frequency response is evaluated at an on-center receiver position and at an exemplary, highly eccentric off-center position located near the boundary of the illuminated area. 
The responses are normalized by the ideal high-frequency value given by \eqref{eq:WFS_amp_dist}, denoted by $\hat{P}$.
All results were computed using a Hann window with the tapering coefficient set to 0.25 and a quasi-continuous discretization ($\Delta s = 1$\,cm)
The amplitude distributions of the integrands are visualized by white fill along the SSD contour, while the resulting effective lengths $\hat{L}_{\mathrm{eff},1/2}$ around the stationary position are highlighted with a red contour, both illustrated for the off-center receiver position.

%For both cases the frequency response of the synthesized field along with the estimated low frequency limits are illustrated in the frontal direction and in an examplary lateral direction, while for the linear SSD case the synthesis is also investigated for a lateral virtual source position.
%For the linear case Fig.~\ref{Fig:Psynth_linear} also presents a test case with lateral virtual source position.
%The frequency response is normalized by it's ideal high frequency value, given by \eqref{eq:WFS_amp_dist}.
%The amplitude of the integrands are illustrated by white fill along the SSD contour for the lateral direction example.
%The resulting effective lengths around the stationary position are denoted by a red contour along the SSD.

In symmetrical cases for on-center receivers, the effective lengths on both sides of the stationary point are equal, and the two cutoff frequencies $f_{c,1}$ and $f_{c,2}$ coincide. 
For both linear and circular SSDs, this predicted limit aligns precisely with the transition between the low-frequency roll-off and the high-frequency asymptotic behavior, where the normalized response magnitude first reaches $0~\text{dB}$.
The off-center responses further clarify the physical role of the dual cutoff frequencies:
\begin{itemize}
\item Below $\min \{f_{c,1}, f_{c,2} \}$, the SPA fails completely, and the response exhibits a $+10~\mathrm{dB/dec}$ slope ($\sim \sqrt{\omega}$).
This lower limit may be slightly underestimated in extreme cases where normalization by the stationary point amplitude artificially inflates the effective length due to the stationary point lying in a low-amplitude region relative to the global maximum.
\item Above $\max \{f_{c,1}, f_{c,2} \}$, the phase change exceeds $\pi/4$ on both sides of the stationary point, fulfilling the SPA conditions and allowing the response to reach its asymptotic high-frequency value.
This defines the primary validity range for 2.5D WFS theory.
\item Between these two limits, the phase change is sufficient only on the side with the longer effective length.
This creates a transition band where the integrand is no longer approximately constant, yet the oscillations do not yet fully cancel, resulting in a response slope of approximately $+5~\mathrm{dB/dec}$.
\end{itemize}

\subsection{Cutoff frequency as function of receiver and source position}

In the following, the low-frequency limit is investigated as a function of both receiver position and virtual source position.
The overall cutoff frequency is defined as $f_c = \max \{f_{c,1}, f_{c,2} \}$, representing the frequency above which the SPA holds for the entire contributing aperture.

%Fig.~\ref{Fig:fc_vs_recpos} depicts the spatial variance of the low frequency limit with varying receiver position and with a fixed virtual source model chosen to be a virtual plane wave and a focused/unfocused point source.
\begin{figure} 
  \begin{center}
    \begin{overpic}[width = 0.43\columnwidth ]{figs/Fc2_lin_vs_recpos_pw2.png}
	\scriptsize
	\put(0,0){(a)}
	\end{overpic}
  \hfill
    \begin{overpic}[width = 0.53\columnwidth ]{figs/Fc2_lin_vs_recpos_ps2.png}
	\scriptsize
	\put(0,0){(b)}
	\end{overpic}
  \\ \vspace{2mm}
    \begin{overpic}[width = 0.3\columnwidth ]{figs/Fc_circ_pw2.png}
	\scriptsize
	\put(0,0){(c)}
	\end{overpic}
%
    \begin{overpic}[width = 0.3\columnwidth ]{figs/Fc_circ_ps2.png}
	\scriptsize
	\put(0,0){(d)}
	\end{overpic}
%
    \begin{overpic}[width = 0.38\columnwidth ]{figs/Fc_circ_fps2.png}
	\scriptsize
	\put(0,0){(e)}
	\end{overpic}
  
%%
\end{center}
\caption{Lower frequency limit of 2.5D WFS as a function of receiver position $\vxr$ for a linear SSD segment (a--b) and a circular SSD (c--e). 
Subfigures (a, c) and (b, d) show the cutoff frequency for a virtual plane wave and a divergent point source, respectively, while (e) depicts the focused source case. 
White contour lines indicate the boundaries for specific cutoff frequencies in Hz, highlighting the regions of flat frequency response. 
All the simulations were ran with a Hann window tapering with tapering coefficient of 0.25.}
	\label{Fig:fc_vs_recpos}
\end{figure}
Fig.~\ref{Fig:fc_vs_recpos} depicts the spatial variance of $f_c$ for both linear and circular SSDs with a fixed virtual source model.
Across all configurations, the cutoff frequency is lowest in the on-center region and increases rapidly toward off-center directions: crucially, even within the nominally illuminated zone, $f_c$ can exceed $1~\mathrm{kHz}$ at highly eccentric positions.
The provided contour lines identify the regions where a flat reconstruction is achieved above specific frequencies, validating that the largest bandwidth is achieved in the central region within a certain distance from the SSD, excluding its immediate proximity.
Below the noted cutoff frequency, the frequency response rolls off with a slope of $-10~\mathrm{dB/dec}$ for on-center positions. 
For off-center positions, the response varies randomly within the transition band (between $\min \{f_{c,1}, f_{c,2}\}$ and $\max \{f_{c,1}, f_{c,2}\}$), exhibiting an average decay of approximately $-5~\mathrm{dB/dec}$.
While this roll-off often occurs within the frequency range typically handled by a subwoofer in satellite WFS deployments, at eccentric positions the transition extends into the mid-frequency range.
In such cases, the resulting spectral tilt and diffraction-induced coloration may become audible, necessitating spectral compensation to maintain timbre consistency across the listening area.

%Across both SSD types, $f_c$ is smallest in the frontal region and increases rapidly toward lateral directions; even inside the illuminated region it can exceed $1~\mathrm{kHz}$. 
%In the transition band between the two limits $f_{c,1}$ and $f_{c,2}$, the synthesized response is typically $4$–$5$~dB down (cf.\ Figs.~\ref{Fig:Psynth_linear} and \ref{Fig:Psynth_circular}), which can be audible as low-frequency coloration.

\begin{figure} 
  \begin{center}
      \begin{overpic}[width = 0.44\columnwidth ]{figs/Fc2_lin_vs_sourcepos.png}
	\scriptsize
	\put(0,0){(a)}
	\end{overpic}
%  \hfill
    \begin{overpic}[width = 0.51\columnwidth ]{figs/Fc2_lin_vs_sourcepos_cross.png}
	\scriptsize
	\put(0,0){(b)}
	\end{overpic}
  \\ \vspace{2mm}
  %
    \begin{overpic}[width = 0.33\columnwidth ]{figs/Fc2_circ.png}
	\scriptsize
	\put(0,0){(c)}
	\end{overpic}
  \hfill
    \begin{overpic}[width = 0.65\columnwidth ]{figs/Fc2_circ_crosssec.png}
	\scriptsize
	\put(0,0){(d)}
	\end{overpic}
\end{center}
  \caption{Lower frequency limit $f_c$ of 2.5D WFS as a function of the virtual source position $\vxs$ for a linear SSD segment (a--b) and a circular SSD (c--d). 
(a, c) 2D distribution of the cutoff frequency; the color scale denotes $f_c$ in Hz (logarithmic) as defined in Fig.~7. 
The vertical dashed lines mark the on-center cross-sections evaluated in the right-hand plots.
(b, d) Cutoff frequency evaluated along the on-center axis for varying SSD sizes ($L_{\mathrm{SSD}}$ or $R_{\mathrm{SSD}}$).
All results utilize a Hann window tapering coefficient of 0.25; in (d), the dashed line additionally illustrates the upward shift of $f_c$ when the tapering coefficient is increased to 0.5.}
	\label{Fig:fc_vs_vspos}
\end{figure}

As a critical factor in the synthesis of dynamic sound fields, such as those produced by moving sources, Fig.~\ref{Fig:fc_vs_vspos} illustrates the dependency of the lower cutoff frequency on the virtual source position $\vxs$. 
For these simulations, the receiver position is fixed at the on-center position for the linear array and at the center for the circular array.
In both geometries, the most severe spectral degradation occurs as the virtual source approaches the SSD or the receiver position. 
In these cases, the cutoff frequency increases rapidly—and theoretically without bound—resulting in an audible loss of low-frequency content. 

At comparable array scales ($L_{\mathrm{SSD}} \approx 2R_{\mathrm{SSD}}$), a circular SSD typically yields a higher low-frequency limit for focused sources, whereas a linear SSD is more favorable for distant sources. 
In a circular geometry, the effective window length is maximized for focused sources because the driving amplitude remains significant along the entire active arc. 
Conversely, for a virtual plane wave, the gradient of the target field vanishes near the array rim, which shortens the effective aperture and raises $f_c$. 
This situation reverses for a linear SSD: a plane wave produces a nearly constant driving amplitude along the segment (yielding a long effective window), while for sources close to the SSD, the driving amplitude decays strongly toward the array ends. 
Additionally, for focused sources near a linear SSD, the phase of the driving function varies more slowly along the array, further reducing the effective window and raising the cutoff frequency.


\section{Conclusion}

This paper established explicit low-frequency validity limits for 2.5D WFS by linking the 2.5D Kirchhoff approximation to the stationary phase approximation (SPA) through the concept of an \emph{effective window length} $\hat{L}_{\mathrm{eff}}$. 
This approach compresses the complex impact of SSD size, tapering, referencing, and source/receiver proximity into a single geometric quantity. 
We derived a closed-form cutoff condition that yields two partial cutoff frequencies, $f_{c,1}$ and $f_{c,2}$ determined by the effective lengths measured from the stationary point; the overall validity limit is \(\max\{f_{c,1},f_{c,2}\}\). 
For on-center directions, the result recovers classical Fresnel theory: the SPA becomes valid once the phase increment from the stationary point to the effective window edge reaches $\pi/4$ (corresponding to a quarter-wavelength across the effective window), which is equivalent to a Fresnel number of $N_F = 0.25$.

The analysis demonstrates that the cutoff frequency is governed by a dependency on the effective distance $d_{\mathrm{rec}}$ and the square of the effective window length $\hat{L}_{\mathrm{eff}}$, where the former limits the bandwidth at large distances and the latter dominates in cases of source or receiver proximity.
The cutoff rises steeply for lateral listening positions and for sources or receivers in close proximity to the SSD. 
Between the two partial limits, a transition band appears exhibiting an average decay of $-5$\,dB/dec, while below the lower limit, the response falls off at a rate of $-10$\,dB/dec.

These results provide the first complete theoretical framework for the low-frequency behavior of WFS, defining the transition between the high-frequency plateau and the low-frequency roll-off. 
While $f_c$ is found to be around $100$\,Hz for central listeners—a range typically covered by subwoofers in practical satellite WFS deployments—our spatial maps identify the boundaries where this limit climbs into the audible mid-range ($>500$\,Hz). 
These findings provide a map for spectral consistency: identifying regions where a flat response is guaranteed and highlighting extreme cases, such as sources in the immediate proximity of the SSD, where audible coloration is inevitable.

Finally, the audibility of the low-frequency loss—especially for lateral conditions—was confirmed by informal listening. 
The listening test can be \emph{reproduced} with our open-source framework: MATLAB scripts for the figures are available at \href{https://github.com/gfirtha/ieee2025b_firtha/tree/gfirtha/MATLAB}{\uline{\small{https://github.com/gfirtha/ieee2025b\_firtha/tree/gfirtha/MATLAB}}}, and real-time binaural rendering for interactive exploration is provided by \href{https://github.com/gfirtha/binauralizer}{\uline{\small{https://github.com/gfirtha/binauralizer}}}.  

{\appendices
\section{A local SPA validity criterion}
\label{app:local_SPA}
As a simple local validity condition for the SPA, it may be required that the leading-order term dominates the second term in the asymptotic expansion of the integral around the stationary position.
Note that in this section, all derivatives are taken with respect to the arc length $s$, with the subscripts omitted for the sake of brevity.
From \eqref{eq:SPA_series}, this condition is expressed as:
\begin{equation}
  |I_0(k)| \geq \left| \frac{I_1(k)}{  \ti k |\phi^{''}(s^*)|} \right|.
\end{equation}
The amplitude of the second term in the expansion is given by \cite{firtha_hahn_schultz_fiala_2025, torresani2000analyse}
{\fontsize{8pt}{\baselineskip}
\selectfont
\begin{multline}
  I_1(k) = 
  I_0(k) \left(\frac{1}{2}\frac{A^{''}(s^*)}{A(s^*)}  \right. - \\ \left. -  \frac{ \frac{1}{8}  \phi^{(4)}(s^*) + \frac{1}{2} \phi^{'''}(s^*)A'(s^*)/A(s^*)}{\phi^{''}(s^*)} + \frac{\frac{5}{24} \phi^{'''}(s^*)^2 }{\phi^{''}(s^*)^2} \right).
  \label{eq:SPA_I1}
\end{multline}}
yielding the general frequency condition:
\begin{equation}
  \scriptsize
  k \geq \frac{A^{''}(s^*)}{2A(s^*)|\phi^{''}(s^*)|}  -  \frac{ \frac{1}{4}  \phi^{(4)}(s^*) +  \phi^{'''}(s^*)A'(s^*)/A(s^*)}{2\phi^{''}(s^*)^2} + \frac{\frac{5}{24} \phi^{'''}(s^*)^2 }{\phi^{''}(s^*)^3}.
\end{equation}

This criterion provides a local estimate of the SPA's validity by evaluating the relative variation of the amplitude and phase functions at the stationary position.
For frontal listener positions, the odd derivatives vanish, and the condition simplifies to
\begin{equation}
  \small
  k \geq \frac{1}{2}\frac{A^{''}(s^*)}{A(s^*)|\phi^{''}(s^*)|}  -  \frac{1}{8} \frac{  \phi^{(4)}(s^*)}{\phi^{''}(s^*)^2}.
\end{equation}
In the proximity of the SSD, the first term typically dominates the right-hand side.
Physically, this implies that for the SPA to be valid, the phase curvature must significantly exceed the curvature of the amplitude distribution. 
This ensures that phase oscillations are rapid enough to cause mutual cancellation everywhere except in the immediate vicinity of the stationary point.

This local condition is particularly relevant for configurations where the integration path is significantly larger than the source and receiver distances, leading to rapid amplitude fluctuations.
But since the involved functions are investigated around the stationary position, therefore, the physical size of the SSD is not taken into consideration, dominating in case of practical WFS geometries.

\section{Phase increment to arbitrary arc length}
\label{app:phase_increment}

We aim to estimate the phase increment along the SSD contour from the stationary position $\vxo^*$ up to an arbitrary arc length increment $\Delta s$.
The phase function $\phi^{P\cdot G}$ defined in \eqref{eq:wfs_phase} can be expanded into a Taylor series around the stationary point $\vxo(s^*)$, reading:
{\fontsize{8pt}{\baselineskip}
\selectfont
\begin{equation}
  \phi^{P\cdot G}(s) \approx \phi^{P\cdot G}(s^*) + \left( s-s^* \right)\phi^{'P\cdot G}_s(s^*) + \frac{\left( s-s^* \right)^2}{2}\phi^{''P\cdot G}_{ss}(s^*)
  \label{eq:phase_taylors}
\end{equation}}
Note that for the sake of brevity notation $\vxo(s)$ is suppressed.
The phase increment $\Delta \phi^{P\cdot G}(\Delta s)$ relative to the stationary point is then expressed as:
{\fontsize{9pt}{\baselineskip}
\selectfont
\begin{multline} 
  \Delta \phi^{P\cdot G}(\Delta s) =  \phi^{P \cdot G}(s^* + \Delta s ) - \phi^{P\cdot G}(s^*) = \\ =
   \Delta s \, \phi^{'P\cdot G}_s(s^*) + \frac{ \Delta s^2}{2}\phi^{''P \cdot G}_{ss}(s^*).
  \label{eq:phase_increment}
\end{multline}}

The first derivative of the phase function along the contour is given by:
\begin{equation}
  \phi'_{s}(\vxo(s)) = \nabla_{\vxo}\phi^{P\cdot G}(\vxo) \cdot \frac{\partial \vxo(s)}{\partial s} = \hks^P(\vxo) - \hks^G(\vx- \vxo),
  \label{eq:first_deriv}
\end{equation}
where $\partial \vxo(s)/\partial s$ is the unit tangential vector of the SSD.
By definition of the stationary point, the tangential components of the local wavenumbers are equal, $\hks^P(\vxo^*) = \hks^G(\vx-\vxo^*)$, which leads to $\phi'_{s}(s^*) = 0$.

The second derivative of the phase for a general geometry is derived in \cite{Firtha2025_ieee_a} as:
\begin{multline}
  \phi''_{ss}(\vxo(s)) = 
   -\hkn^P(\vxo)^2 \left(\frac{1}{\rho^G_h(\vx-\vxo)} + \frac{1}{\rho^P_h(\vxo)} \right) - \\
   -2 \left( \hks^P(\vxo) - \hks^G(\vx-\vxo)\right) \kappa_{\mathrm{SSD}}(\vxo),
   \label{eq:curvature_horiz}
\end{multline}
where $\kappa_{\mathrm{SSD}}$ is the local curvature of the SSD.
At the stationary position, the second term in \eqref{eq:curvature_horiz} vanishes, simplifying the required phase increment to:
\begin{equation} 
 \Delta \phi^{P\cdot G}(\Delta s) = - \frac{\Delta s^2}{2} \hkn^P(\vxo^*)^2 \left(\frac{1}{\rho^G_h(\vx-\vxo^*)} + \frac{1}{\rho^P_h(\vxo^*)} \right).
 \label{eq:phase_inc}
\end{equation}
By substituting the horizontal distance correction factor $d^{\mathrm{c}}_{h}$ from \eqref{eq:25D_WFS}, this result yields the simplified form utilized in \eqref{eq:phase_inc_main}.


  % The first derivative $\phi^{'P+G}_s(s^*)$ read as
  % {\fontsize{8pt}{\baselineskip}
  % \selectfont
  % \begin{multline}
  %   \phi^{'P+G}_s(\vxo(s)) = \frac{\partial }{\partial s}\phi^{P+G}(\vxo(s)) = \\ =\nabla_{\vx}\phi^{P+G}(\vxo(s)) \cdot \frac{\partial \vxo(s)}{\partial s}
  %   = \hks^P(\vxo) - \hks^G(\vx- \vxo)
  %    \label{eq:first_deriv}
  % \end{multline}}
  % where $\frac{\partial \vxo(s)}{s}$ is the tangential vector of the SSD assuming an arc length parametrization.
  % For the second derivative the derivation is given in Appendix A of \cite{Firtha2025_ieee_a}, with the final result being
  % \begin{multline}
  %   \phi^{P+G''}_{ss}(\vxo(s)) = 
  %    -\hat{k}_{\mathrm{n}}(\vxo)^2 \left(\frac{1}{\rho^G_h(\vx-\vxo)} + \frac{1}{\rho^P_h(\vxo)} \right) - \\
  %    -2 \left( \hat{k}^P_{\mathrm{s}}(\vxo) - \hat{k}^G_{\mathrm{s}}(\vx-\vxo)\right) \kappa_{\mathrm{SSD}}(\vxo),
  %    \label{eq:curvature_horiz}
  % \end{multline}
  % where $\kappa_{\mathrm{SSD}}$ is the local curvature of the SSD.
  % By definition, in the stationary position $\hat{k}^P_{\mathrm{s}}(\vxo) = \hat{k}^G_{\mathrm{s}}(\vx-\vxo)$ holds, simplifying both \eqref{eq:first_deriv} and \eqref{eq:curvature_horiz} as
  % \begin{align}
  %    \phi^{'P+G}_s(\vxo(s^*)) &= 0,\\
  %     \phi^{P+G''}_{ss}(\vxo(s^*)) &= -\hat{k}_{\mathrm{n}}(\vxo)^2 \left(\frac{1}{\rho^G_h(\vx-\vxo)} + \frac{1}{\rho^P_h(\vxo)} \right) 
  % \end{align}
  % Finally, the required phase increment is given from \eqref{eq:phase_increment} by
  % {\fontsize{9pt}{\baselineskip}
  % \selectfont
  % \begin{multline} 
  %  \Delta \phi^{P+G}(\Delta s) = \frac{ \Delta s ^2}{2}\phi^{''P+G}_{ss}(s^*) = \\ =
  % - \frac{ \Delta s ^2}{2} \hat{k}_{\mathrm{n}}(\vxo^*)^2 \left(\frac{1}{\rho^G_h(\vx-\vxo^*)} + \frac{1}{\rho^P_h(\vxo^*)} \right).
  % \label{eq:phase_inc}
  % \end{multline}}
}

\section*{Acknowledgments}
This work was supported by the János Bolyai Research Scholarship of the Hungarian Academy of Science and by the OTKA PD-143129 and OTKA K-143436 grants.

%
\appendix[Phase increment at the SSD endpoint]
\paragraph{Virtual plane wave with no tapering applied}
In the following the special case of a virtual plane wave, synthesized by a finite SSD with the length of $L_x^{\mathrm{SSD}}$ is investigated.
The plane wave's propagation direction is perpendicular to the SSD ($k^P_x = 0, \, k^P_y = k$).
The synthesis is referenced to a reference line a along $\yref$, and the synthesized field is investigated in the frontal reference position $\vxref = [0,\, \yref,\, 0]$.
In this simple setup the synthesized field reads as 
\begin{equation}
  P(\vxref, \omega) = \int_{-L_x/2}^{L_x/2} \overbrace{\sqrt{ 8 \pi \ti k \yref }}^{D(\vxo, \omega)} \frac{\te^{- \ti k |\vxref - \vxo|}}{4 \pi |\vxref - \vxo|}  \td x_0,
\end{equation}
}
with $|\vxref - \vxo| = \sqrt{ x_0^2 + \yref^2 }$.
Since the amplitude of the target field at the reference position is unity, therefore, the cut-off wavenumber for the far-field approximation is simply given as
\begin{equation}
  k_{\mathrm{FF}} = \frac{1}{ \left( \int_{-L_x/2}^{L_x/2} \sqrt{ \frac{ \yref }{2 \pi}} \frac{1}{ |\vxref - \vxo|}  \td x_0 \right)^2}.
\end{equation}
For $\yref \gg L_x/2$ the amplitude changes only slightly over the integral path and the involved integral can be simply approximated as
\begin{equation}
  \small
  \int_{-L_x/2}^{L_x/2} \frac{1}{ \sqrt{x_0^2 + \yref^2}}  \td x_0 = \frac{1}{\yref} \int_{-L_x/2}^{L_x/2} \frac{1}{ \sqrt{ \underbrace{\left(\frac{x_0}{\yref}\right)^2}_{\ll 1} + 1}}  \td x_0 \approx \frac{L_x}{\yref},
\end{equation}
resulting in the cutoff wavenumber
\begin{equation}
  k_{\mathrm{FF}} \approx 2 \pi \frac{\yref}{ L_x^2}.
\end{equation}

Now we are interested about the phase variation of the integrand on this cutoff frequency.
In this frontal case the phase function is an even function of $x_0$, given merely by the phase of the Green's function $-k |\vxref - \vxo|$.
The phase increment measured from the center of the SSD to each end of the SSD at the cutoff wavenumber is given by
\begin{equation}
  \Delta \phi = k_{\mathrm{FF}} \left( \sqrt{\left(\frac{L_x}{2}\right)^2 + \yref^2} -  \yref \right).
  \label{eq:deltaPhi}
\end{equation}
Again, in case of $L_x \ll \yref$, distance of the reference point and the SSD endpoint can be expanded into a binomial (MacLaurin) series, which up to the leading order reads as
\begin{multline}
  \yref \sqrt{\left(\frac{L_x}{2 \yref}\right)^2 + 1} \approx \\ \approx \yref \left( 1 + \frac{1}{2} \left(\frac{L_x}{2 \yref}\right)^2 - \frac{1}{8}  \left(\frac{L_x}{2 \yref}\right)^4 + ...\right) = \\
  = \yref  + \frac{L_x^2}{8 \yref}. 
\end{multline}
Substituting back to \eqref{eq:deltaPhi} gives us the phase increment
\begin{equation}
  \Delta \phi = k_{\mathrm{FF}} \left( \frac{L_x^2}{8 \yref} \right) = 2 \pi \frac{\yref}{ L_x^2} \, \frac{L_x^2}{8 \yref} = \frac{\pi}{4}.
\end{equation}
This equation states that independently of the actual receiver position, in the far field of the SSD the phase change towards the SSD endpoint at the cutoff frequency/wavenumber is identically given by $\frac{\pi}{4}$.

\paragraph{Virtual plane wave with tapering}
As a more complicated extension let's assume an arbitrary tapering window applied to the driving functions $w(x_0)$, resulting in the synthesized field
\begin{equation}
  P(\vxref, \omega) = \int_{-L_x/2}^{L_x/2} \overbrace{ w(x_0)\sqrt{ 8 \pi \ti k \yref }}^{D(\vxo, \omega)} \frac{\te^{- \ti k |\vxref - \vxo|}}{4 \pi |\vxref - \vxo|}  \td x_0,
\end{equation}
Furthermore, let's introduce an effective SSD length defined as
\begin{equation}
  L_{x,\mathrm{eff}} = \int_{-\frac{Lx}{2}}^{\frac{L_x}{2}} w(x_0) \td x0!
\end{equation}
By following the same derivation as above the farfield cut-off frequency reads as
\begin{equation}
  k_{\mathrm{FF}} \approx 2 \pi \frac{\yref}{ L_{x,\mathrm{eff}}^2}.
\end{equation}
Finally, investigating the phase increment up to the point $L_{x,\mathrm{eff}}/2$ yields
\begin{equation}
  \Delta \phi = k_{\mathrm{FF}} \left( \sqrt{\left(\frac{L_{x,\mathrm{eff}}}{2}\right)^2 + \yref^2} -  \yref \right) \approx \frac{\pi}{4}.
\end{equation}
Hence, when a slowly changin tapering is applied to the driving functions, when the receiver is sufficiently far from the SSD, the phase change at the end of the effective length equals $\pi/4$ independently of the actual source position. 

\paragraph{Virtual point source}
Finally, assume a virtual point source located at $\vxs = [0,\, -\ys,\, 0]$.
The synthesized field with applying tapering, again, with referencing the synthesis to a parallel reference line is given by
\begin{multline}
  P(\vxref, \omega) = \sqrt{\ti k}  \frac{\ys}{4\pi\sqrt{2\pi}}\sqrt{  \frac{\yref}{\yref+\ys}} \\ \int_{-L_x/2}^{L_x/2} w(x_0)\frac{\te^{- \ti k |\vxo - \vxs|}}{ |\vxo-\vxs|^{3/2}}   \frac{\te^{- \ti k |\vxref - \vxo|}}{ |\vxref - \vxo|}  \td x_0,
\end{multline}
Now the amplitude of the integrand is approximated as 
\begin{multline}
  \small
  \int_{-L_x/2}^{L_x/2} \frac{w(x_0)}{ \sqrt{x_0^2 + \ys^2}^{3} \sqrt{x_0^2 + \yref^2}}  \td x_0 = \\
  \frac{1}{\ys^{3/2}\yref} \int_{-L_x/2}^{L_x/2} \frac{w(x_0)}{ \sqrt{ \left(\frac{x_0}{\ys}\right)^2 + 1}^{3/2} \sqrt{ \left(\frac{x_0}{\yref}\right)^2 + 1}}  \td x_0 \\ 
  \approx \frac{L_{x,\mathrm{eff}}}{\ys^{3/2}\yref},
\end{multline}
and the synthesized field at $\omega = 0$ is approximated as
\begin{equation}
  \scriptsize
  P(\vxref, 0) = \sqrt{\ti k}  \frac{L_{x,\mathrm{eff}}}{4\pi\sqrt{2\pi}}\sqrt{  \frac{1}{\ys\yref (\yref+\ys)}}.
\end{equation}
The target field in the present setup is given by
\begin{equation}
  P_{\mathrm{target}}(\vxref, 0) = \frac{1}{4\pi}\frac{1}{\yref+\ys},
\end{equation}
hence the synthesized field normalized by the target field is given by
\begin{equation}
  \scriptsize
  \hat{P}(\vxref, 0) = \sqrt{\ti k}  \frac{L_{x,\mathrm{eff}}}{\sqrt{2\pi}}\sqrt{  \frac{\yref+\ys}{\ys\yref }}.
\end{equation}
Again, the cutoff wavenumber is, therefore, approximated as
\begin{equation}
  k_{\mathrm{FF}} \approx \frac{2\pi}{L_{x,\mathrm{eff}}^2} \frac{\ys \, \yref}{\yref+\ys}.
  \label{eq:k_cutoff_ps}
\end{equation}

The phase change up to the endpoint of the SSD reads as
\begin{equation}
  \scriptsize
  \Delta \phi = k_{\mathrm{FF}} \left( \sqrt{\left(\frac{L_{x,\mathrm{eff}}}{2}\right)^2 + \yref^2} + \sqrt{\left(\frac{L_{x,\mathrm{eff}}}{2}\right)^2 + \ys^2} -  \yref - \ys \right).
\end{equation}
which by using its MacLaurin series can be written by 
\begin{equation}
  \Delta \phi = k_{\mathrm{FF}} \left( \frac{L_{x,\mathrm{eff}}^2}{8 \yref} + \frac{L_{x,\mathrm{eff}}^2}{8 \ys}  \right) =  k_{\mathrm{FF}} \frac{L_{x,\mathrm{eff}}^2}{8}\left(  \frac{\yref+\ys}{\yref \ys}  \right) 
\end{equation}
Finally, by substituting back \eqref{eq:k_cutoff_ps}
\begin{equation}
  \Delta \phi =  \frac{2\pi}{L_{x,\mathrm{eff}}^2} \frac{\ys \, \yref}{\yref+\ys} \frac{L_{x,\mathrm{eff}}^2}{8}\left(  \frac{\yref+\ys}{\yref \ys}  \right) = \frac{\pi}{4}.
\end{equation}
Again, it is verified analytically that in case of the synthesis of a virtual point source when the receiver and the virtual source is sufficiently far from the SSD (further than the effective SSD length) the cutoff frequency is given by that point where the phase change at the SSD's endpoint equals to $\pi/4$, i.e. the SSD contains the quater of the wavelength.

%\paragraph{Arbitrary virtual source}
\bibliographystyle{IEEEtran}
\bibliography{dissertation}

\end{document}


