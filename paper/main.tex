\documentclass[lettersize,journal]{IEEEtran}
\usepackage{amsmath,amsfonts}
\usepackage{algorithmic}
\usepackage{algorithm}
\usepackage{array}
\usepackage[caption=false,font=normalsize,labelfont=sf,textfont=sf]{subfig}
\usepackage{textcomp}
\usepackage{stfloats}
\usepackage{url}
\usepackage{verbatim}
\usepackage{graphicx}
\usepackage{cite}
\usepackage{overpic}
\usepackage{xcolor}
\usepackage{hyperref}
\hypersetup{
    colorlinks=true,
    linkcolor=blue,
    filecolor=blue,      
    urlcolor=blue,
    citecolor=black,
    pdftitle={Overleaf Example},
    pdfpagemode=FullScreen,
    }
    
\usepackage[normalem]{ulem}
\hyphenation{op-tical net-works semi-conduc-tor IEEE-Xplore}
% updated with editorial comments 8/9/2021
\usepackage{amsmath, amssymb, bm}

\newcount\posveccount
\newcommand*\posvec[1]{
        \global\posveccount#1
        [
        \posvecnext
}
\def\posvecnext#1{
        #1
        \global\advance\posveccount-1
        \ifnum\posveccount>0
                ,\
                \expandafter\posvecnext
        \else
                ]^{\mathrm{T}}
        \fi
}

\newcount\colveccount
\newcommand*\colvec[1]{
        \global\colveccount#1
        \begin{bmatrix}
        \colvecnext
}
\def\colvecnext#1{
        #1
        \global\advance\colveccount-1
        \ifnum\colveccount>0
                \\[5pt]
                \expandafter\colvecnext
        \else
                \end{bmatrix}
        \fi
}


\newcommand{\dint}{\int\!\!\!\!\!\int}
\newcommand{\tint}{\int\!\!\!\!\int\!\!\!\!\int}
\newcommand{\qint}{\int\!\!\!\!\int\!\!\!\!\int\!\!\!\!\int}
\newcommand{\td}{\mathrm{d}}
\newcommand{\te}{\mathrm{e}}
\newcommand{\ti}{\mathrm{j}}
\newcommand{\sinfi}{\sin\varphi}
\newcommand{\cosfi}{\cos\varphi}
\newcommand{\sinteta}{\sin\theta}
\newcommand{\costeta}{\cos\theta}
\newcommand{\yref}{y_{\mathrm{ref}}}
\newcommand{\ys}{y_{\mathrm{s}}}
\newcommand{\dref}{d_{\mathrm{ref}}}
\newcommand{\vx}{\mathbf{x}}
\newcommand{\vxo}{\mathbf{x}_0}
\newcommand{\vxc}{\mathbf{x}_{\mathrm{c}}}
\newcommand{\vxs}{\mathbf{x}_{\mathrm{s}}}
\newcommand{\vxr}{\mathbf{x}_{\mathrm{ref}}}
\newcommand{\vk}{\mathbf{k}}
\newcommand{\nn}{\mathrm{n}}
\newcommand{\vn}{\mathbf{n}}
\newcommand{\vnin}{\mathbf{n}_{\mathrm{in}}}
\newcommand{\vhk}{\hat{\mathbf{k}}}
\newcommand{\kn}{k_\mathrm{n}}
\newcommand{\hkn}{\hat{k}_\mathrm{n}}
\newcommand{\hks}{\hat{k}_\mathrm{s}}
\newcommand{\vxref}{\mathbf{x}_{\mathrm{ref}}}
\newcommand{\phix}{\phi'_{x}}
\newcommand{\phixx}{\phi''_{xx}}

\newcommand{\rp}{r_{{\tiny \mathrm{P}}}}
\newcommand{\rg}{r_{\mathrm{G}}}
\newcommand{\rref}{r_{\mathrm{ref}}}

\newcommand{\phiy}{\phi'_{y}}
\newcommand{\phiyy}{\phi''_{yy}}

\newcommand{\phiz}{\phi'_{z}}
\newcommand{\phizz}{\phi''_{zz}}

\newcommand{\phiPxx}{\phi^{P''}_{xx}}
\newcommand{\phiGxx}{\phi^{G''}_{xx}}

\newcommand{\phiPx}{\phi^{P'}_{x}}
\newcommand{\phiGx}{\phi^{G'}_{x}}

\newcommand{\phiPyy}{\phi^{P''}_{yy}}
\newcommand{\phiGyy}{\phi^{G''}_{yy}}

\newcommand{\phiPzz}{\phi^{P''}_{zz}}
\newcommand{\phiGzz}{\phi^{G''}_{zz}}  
\newcommand{\Psynth}{P_{\mathrm{synth}}}  
\newcommand{\Pdiffr}{P_{\mathrm{diffr}}}  
\newcommand{\Pfoc}{P_{\mathrm{foc}}}  
\newcommand{\Pfocsynth}{P_{\mathrm{synth,foc}}}  
\newcommand{\Rssd}{r_{\mathrm{SSD}}}  
\newcommand{\Leff}{\hat{L}_{\mathrm{eff}}} 



\newcommand{\mH}{\mathbf{H}}



\usepackage{amssymb}

\let\oldemptyset\emptyset
\let\emptyset\varnothing

\begin{document}

\title{Limitations of Wave Field Synthesis Part II:\\low-frequency limits}

\author{Gergely Firtha
\thanks{Manuscript received April 19, 2021; revised August 16, 2021.}}


% The paper headers
\markboth{Journal of \LaTeX\ Class Files,~Vol.~14, No.~8, August~2021}%
{Shell \MakeLowercase{\textit{et al.}}: A Sample Article Using IEEEtran.cls for IEEE Journals}

\IEEEpubid{0000--0000/00\$00.00~\copyright~2021 IEEE}
% Remember, if you use this you must call \IEEEpubidadjcol in the second
% column for its text to clear the IEEEpubid mark.

\maketitle

\begin{abstract}
\end{abstract}


\begin{IEEEkeywords}
Article submission, IEEE, IEEEtran, journal, \LaTeX, paper, template, typesetting.
\end{IEEEkeywords}

\section{Introduction}

Wave Field Synthesis (WFS) is an advanced spatial audio rendering technique capable of accurately synthesizing sound fields by employing a dense array of secondary loudspeakers distributed along a contour or surface, known as the secondary source distribution (SSD).

Originally introduced by Berkhout in the late 1980s \cite{Berkhout1988, doi:10.1121/1.404755, Berkhout1993:Acoustic_control_by_WFS}, WFS utilizes the Huygens–Fresnel principle to reconstruct intended acoustic wavefronts over an extended listening area. Over subsequent decades, significant theoretical advancements have refined its mathematical framework \cite{Vogel1993:phd, Start1997:phd, Verheijen1997:phd, Sonke1998}, expanded its scope to include innovative virtual source paradigms such as focused and moving sources \cite{Spors2009focused, Wierstorf2013focused, Oldfield2013_FocusedWFS, Ahrens2008moving, doi:10.1121/1.4996126}, and systematically addressed various implementation constraints.

Despite these advances, WFS inherently possesses limitations, particularly at the lower end of its operational frequency range. The integral representations used—specifically the Kirchhoff approximation and the Rayleigh integral—are fundamentally high-frequency approximations. Consequently, low-frequency performance deteriorates below a geometry-dependent threshold \cite{Spors10ahrens:analysis}. Although this phenomenon is acknowledged within the literature, a comprehensive analytical treatment and rigorous understanding of the precise low-frequency limits remain largely unexplored.

This paper, as the second part of a two-part series, specifically addresses these gaps by rigorously examining the low-frequency behavior and limits inherent to WFS. We analyze how the intrinsic approximations influence low-frequency wavefront reproduction and derive explicit criteria and characterizations for assessing low-frequency performance. Moreover, we extend our discussion to focused sources, a case exhibiting pronounced low-frequency artifacts due to truncation and tapering effects, areas previously under-examined in existing literature.

Following the principles of reproducible research, the MATLAB code for the figures in this article is publicly accessible at \href{https://github.com/gfirtha/ieee2025b_firtha/tree/gfirtha/MATLAB}{\uline{\small{https://github.com/gfirtha/ieee2025b\_firtha/tree/gfirtha/MATLAB}}}.

\section{Theoretical basics}

\subsection{Problem formulation}
Consider an arbitrary three-dimensional sound field expressed generally in polar form as
\begin{equation}
P(\vx,\omega) = A^P(\vx) , \te^{\ti k \phi^P(\vx)},
\label{eq:general_field}
\end{equation}
where the position vector $\vx = \posvec{3}{x}{y}{z}$ identifies any arbitrary point in 3D space, $k = \frac{\omega}{c}$ represents the acoustic wavenumber, and $c$ denotes the speed of sound.

Wavefronts are surfaces characterized by constant phase (isophase surfaces), given by $\phi^P = \mathrm{constant}$. 
These can be locally described up its second order Taylor's approximation by their propagation direction and curvature when viewed within an intrinsic coordinate system aligned with the ray direction.

We define the normalized local wavenumber vector as
\begin{equation}
\vhk^P(\vx) = -\nabla_{\vx} \phi^P(\vx), \hspace{5mm} |\vhk^P(\vx)| = 1,
\label{eq:eikonal}
\end{equation}
which is a unit vector oriented towards the direction of maximum phase progression, thus indicating the local direction of propagation. 
The curvature of the wavefront is described by the second derivative of the phase as
\begin{equation}
\kappa^P_{1,2}(\vx) = \frac{1}{\rho^P_{1,2}(\vx)} = - \mathrm{eigs}\left( \frac{\partial^2 \phi^P(\vx)}{\partial \vx_i \partial \vx_j} \right).
\end{equation}

In polar form, the gradient of the sound field is expressed as
\begin{multline}
\nabla_{\vx} P(\vx) = P'(\vx) = \left(\frac{\nabla_{\vx} A^P(\vx)}{A^P(\vx)} + \ti k \nabla_{\vx} \phi^P(\vx) \right)P(\vx) = \\
= \left(\overline{\kappa}^P(\vx) + \ti k \right) \vhk^P(\vx) P(\vx),
\label{eq:gradient}
\end{multline}
where $\overline{\kappa}^P(\vx)$ signifies the mean curvature of the wavefront.
In the far field of the source of the sound $|\overline{\kappa}^P(\vx)| \ll k$ holds and the gradient can be approximated as
\begin{equation}
\nabla_{\vx} P(\vx) \approx  \ti k \vhk^P(\vx) P(\vx),
\label{eq:pw_appr}
\end{equation}
coined here as a local plane wave approximation, since the equation holds with equality for the case of a plane wave ($\overline{\kappa}^P(\vx) = 0$).

In sound field synthesis, the goal is to replicate this specified wavefront using a continuous array of secondary point sources arranged along a contour $C$, known as the secondary source distribution (SSD). These secondary sources are driven by functions $D(\vxo,\omega)$, leading to the synthesized sound field:
\begin{equation}
\Psynth(\vx,\omega) = \oint_C D(\vxo, \omega) G(\vx-\vxo) \td s(\vxo).
\label{eq:synth_field}
\end{equation}
The synthesized field is intended to precisely match the target field. Here, $G(\vx-\vxo) = \frac{1}{4\pi}\frac{\te^{-\ti k|\vx-\vxo|}}{|\vx-\vxo|}$ is the 3D free-field Green's function, describing the radiation from a secondary point source, and $s(\vxo)$ denotes the arc length along the SSD.

The contour of these secondary sources is positioned in a horizontal plane, defined at $z=0$, aligned with the listener's ear level, known as the synthesis plane. 
The associated geometry is depicted in Figure \ref{Fig:geometry}.


\begin{figure} 
  \begin{center}
    \begin{overpic}[width = 1\columnwidth ]{figs/WFS_geometry_2b.png}
	\scriptsize
	\put(61.5,40.5){$\vxo$}
	\put(44,40){$\vxref(\vxo)$}
	\put(24,35){$\vx$}
	\put(3,19){synthesis plane}
	\put(40,22){$C$: SSD contour}
	\put(8,50){$S$: integral surface}
	\put(80.5,71){reference curve}
	\put(80.5,61.5){active SSD}
	\put(56,34){$\vhk^P(\vxo)$}
	\put(84,47){$x$}
	\put(95,29){$y$}
	\put(97,58){$z$}
	\put(81,42){$\vxs$}
	%\put(72,37){$\rho^P(\vxo)$}
  \put(80,66){{\fontsize{7pt}{\baselineskip}
  \selectfont \parbox{4cm}{vertical integral\\path}  }}
	\end{overpic}
\end{center}
\caption{Geometry for deriving 2.5D Kirchhoff approximation and WFS driving functions.}
	\label{Fig:geometry}
\end{figure}

\subsection{The 3D Kirchhoff-approximation}
The starting point of WFS theory involving arbitrary SSD geometries is the simple source formulation of the target sound field inside the listening region, expressed in term of a single layer potential.
For a closed surface $\Omega$ that encloses this region one has \cite{colton1983integral,ColtonKress1998}
{\fontsize{10pt}{\baselineskip}
\selectfont
\begin{equation}
  \small
  P(\vx) = \oint_\Omega  \underbrace{\left(P'_{\mathrm{n}}(\vxo, \omega) - P'_{\mathrm{scat,n}}(\vxo, \omega)\right)}_{\mu(\vxo)} G(\vx-\vxo) \td S(\vxo).
\end{equation}} 
The term $P_{\mathrm{scat}}$ is the \emph{exterior} pressure field produced when an acoustically soft surface is placed on~$\Omega$ and excited by $P$; see \cite{Fazi2013:Equivalent_scattering}.  
The jump in the pressure gradient $\mu(\mathbf{x}_{0},\omega)$ constitutes the driving signal for the secondary sources.

In the context of WFS the surface of integration is chosen to be $\Omega$ is translationally invariant in the vertical direction; its intersection with the horizontal plane is the actual SSD contour (Fig.~\ref{Fig:geometry}). 
The integration surface is illustrated in Fig.~\ref{Fig:geometry}.
%In the kernel of the integral $P_{\mathrm{scat}}(\vxo, \omega)$ is an exterior field, scattered from a sound soft surface along $\Omega$, on which the scattered pressure is given by $P(\vxo, \omega), \vxo \in \Omega$ \cite{Fazi2013:Equivalent_scattering}. 
%In the integral $\mu(\vxo)$ represents the pressure gradient jump between the interior and exterior sound fields, yielding driving functions for the general synthesis problem.
%The equivalent scattered field is known analytically only in special (planar, spherical, etc.) geometries, while for arbitrary geometries numerical solution is required by using e.g. the Boundary Element Method.
Unfortunately, except for canonical geometries (plane, sphere, cylinder, …) $P'_{\mathrm{scat,n}}$ is not available in closed form, and a numerical Boundary Element Method is required—clearly unsuitable for real-time reproduction.
%The practical applicability of the above formulation is, therefore, limited in the context of sound field reproduction.

Assuming that the integral/scatterer surface can be considered locally planar within the scale of the wavelength the scattered field can by approximated by $P'_{\mathrm{scat,n}}(\vxo, \omega) \approx -P'_{\mathrm{n}}(\vxo, \omega)$, yielding the Kirchoff approximation (KA) or tangent plane approximation of the interior sound field \cite{Pierce1991, Ogilvy1991, FelsenMarcuvitz1994, Chew1995}
{\fontsize{9pt}{\baselineskip}
\selectfont
\begin{equation}
  \small
  P(\vx) \approx \oint_\Omega 2 w(\vxo) \, P'_{\mathrm{n}}(\vxo, \omega) G(\vx-\vxo) \td S(\vxo).
  \label{eq:KA}
\end{equation}}
The window function, given by
\begin{equation}
  w(\vxo) = 
  \begin{cases}
    1  \hspace{5mm} \text{if} \hspace{1cm} \hkn^P(\vxo)\geq 0, \\
    0   \hspace{5mm} \text{otherwise}
  \end{cases}
  \label{eq:win}
\end{equation}
performs the partitioning of the integral surface to an acoustically illuminated and a shadow region.
For an infinite planar surface the above expression holds with equality, termed as the Rayleigh integral.

The Kirchhoff approximation is intrinsically a \emph{high-frequency} model: creeping waves (diffraction) and secondary reflections between different parts of~$\Omega$ are disregarded.  
The general validity of the KA is a complex, extensively researched question, however, with no general analytical formulation available, so far.
The question frequently arises in the aspect of calculating fields scattered from rough, typically non-convex surfaces, in which typically far-field assumptions are made e.g. scattering of plane waves evaluated far from the scatterer surface.
Sufficient validity conditions, normally derived for plane-wave scattering from rough or non-convex surfaces, are \cite{Franco2017,Fung1981,Eaton2006}
\begin{equation}
      2 k \hkn^{P}(\vxo) \overline{\rho}_{\Omega}(\vxo) \gg 1, \text{or} \hspace{5mm} 2 k \hkn^{P}(\vxo)^3 \overline{\rho}_{\mathrm{\Omega}}(\vxo) \gg 1
  \end{equation}
with $\overline{\rho}_{\Omega}$ being the mean local radius of the integral surface.
In the present treatise, however, we are interested about its validity for smooth integral surfaces, but with both the sound source and receiver positions typically located in the proximity of the SSD.
In this scenario it will be shown that the low frequency limit of applicability is mainly dominated by the size of the integral surface (SSD) and the structure of the integral's amplitude distribution ($\mu(\vxo)$).

From the integral formulation \eqref{eq:KA} the 3D WFS driving functions could be directly extraced, allowing the synthesis of the target field applying an enclosing SSD surface.
In order to arrive at driving functions for a contour the following asymptotic approximation is applied to the 3D KA.

\subsection{The Stationary Phase Approximation}

The stationary–phase approximation (SPA) is a classical tool for the asymptotic evaluation of rapidly oscillatory integrals.
Consider a Fourier-type contour integral ass
\begin{equation}
  I(k) = \oint_C A(s) \te^{\ti k \phi(s)} \td s
  \label{eq:SPA_int}
\end{equation}
The approximation relies on the idea that under high frequency conditions rapid oscillations may cancel out and the integral is dominated by positions in the integral path where the phase does not change, termed as the \emph{stationary positions}.
The stationary positions are found where the phase gradient vanishes.
In the geometry relevant to the present work the surface integrals separate into products of one-dimensional integrals; hence it suffices to treat the 1-D case, for which the stationary position is found where
\begin{equation}
  \phi'(s^{\ast}) \;=\; 0, 
  \qquad
  \phi''(s^{\ast})\neq 0
  \label{eq:stationary_point}
\end{equation}
holds.

The derivation of the SPA relies on the Taylor's expansion of both the phase and the amplitude function, performing analytical Gaussian integration termwise and collecting same powers of $k$ together to rewrite the original integral as a power series of $k$ in the form
\begin{equation}
  I(k) = \sum_{i = 0}^{\infty} \frac{I_i(k)}{  \left(\ti k |\Phi^{''}(s^*)|\right)^{-i}}
  \label{eq:SPA_int}
\end{equation}
where $\sim$ denotes equality at $k \rightarrow \infty$.
The leading order (geometrical-optics) term is given by 
{\fontsize{10pt}{\baselineskip}
\selectfont
\begin{equation}
  I_0(k) \sim % = \sum_{i = 0}^{\infty} \frac{I_i}{\left( k |\Phi^{''}(s^*)| \right)^{i+0.5}} \sim \\
  A(s^*) \te^{\ti k \Phi(s^*)} \sqrt{\frac{2\pi}{k |\Phi^{''}(s^*)|}} \te^{\ti \frac{\pi}{4}\mathrm{sign}(\Phi^{''}(s^*))},
  \label{eq:SPA0}
\end{equation}}
termed as the stationary phase approximation of the integral \eqref{eq:SPA_int}, while the second term can be expressed as
{\fontsize{8pt}{\baselineskip}
\selectfont
\begin{multline}
  I_1(k) = 
  I_0(k) \left(\frac{1}{2}\frac{A^{''}(s^*)}{A(s^*)}  \right. - \\ \left. -  \frac{ \frac{1}{8}  \Phi^{(4)}(s^*) + \frac{1}{2} \Phi^{'''}(s^*)A'(s^*)/A(s^*)}{\Phi^{''}(s^*)} + \frac{\frac{5}{24} \Phi^{'''}(s^*)^2 }{\Phi^{''}(s^*)^2} \right).
\end{multline}}

%{\fontsize{9pt}{\baselineskip}
%\selectfont
%\begin{multline}
%  I_0(k) \sim % = \sum_{i = 0}^{\infty} \frac{I_i}{\left( k |\Phi^{''}(s^*)| \right)^{i+0.5}} \sim \\
%  \overbrace{A(s^*) \te^{\ti k \Phi(s^*)} \sqrt{\frac{2\pi}{k |\Phi^{''}(s^*)|}} \te^{\ti \frac{\pi}{4}\mathrm{sign}(\Phi^{''}(s^*))}}^{I_0} \cdot \\ \cdot \left(1 + \frac{I_1}{\ti k |\Phi^{''}(s^*)|} +  \frac{I_2}{ ( \ti  k |\Phi^{''}(s^*)|)^2} + ... +  \frac{I_i}{ (\ti  k |\Phi^{''}(s^*)|)^i}  \right)
%  \label{eq:fullSPA}
%\end{multline}}
%As an example, the first non-leading term is given by
%\begin{equation}
%  \footnotesize
%  I_1 = 
%  \frac{1}{2}\frac{A^{''}(s^*)}{A(s^*)} - \frac{ \frac{1}{8}  \Phi^{(4)}(s^*) + \frac{1}{2} \Phi^{'''}(s^*)A'(s^*)/A(s^*)}{\Phi^{''}(s^*)} + \frac{\frac{5}{24} \Phi^{'''}(s^*)^2 }{\Phi^{''}(s^*)^2},
%\end{equation}
%and where each term $I_i$ can be written as a further power series of $\frac{1}{\Phi^{''}(s^*)}$.
%The leading term $I_0$ is the usual form of the stationary phase approximation of the integral, emerging from the Taylor's approximation of the ampltideu and phase up to the leading and the second order respectively.
%Thus, the integral of the function is approximated by its stationary value along with a curvature correction term.

The low frequency validity of the above approximation is of central importance of the present contribution, and will be discussed in the following sections when the formula is applied to integral formulations of sound fields. 

\subsection{The 2.5D Kirchhoff-approximation}
The geometry for the Kirchhoff approximation (Eq.~\ref{eq:KA}) is shown in \ref{Fig:geometry}, with the window function \eqref{eq:win} denoted by pale red surface.
Due to the vertical invariancy the integral can be split to a vertical and a horizontal integral ($\int_\Omega ... \td S(\vxo) = \int_{C} \int_{-\infty}^{\infty} ... \td z_0 \td s(x_0,y_0)$).
In WFS theory the SPA is applied to the Kirchhoff approximation along the vertical dimension in order to reduce the surface integral to a contour integral along the phyisically realizable SSD contour.

The stationary position for the entire surface integral is found where the gradient of the integrand's phase function vanishes, i.e. where $\vhk^P(\vxo^*) = \vhk^G(\vx-\vxo^*)$ holds (the local propagation direction of the Green's function is aligned with that of the virtual field along the SSD).
For virtual fields located in the the synthesis plane with $\hat{k}^P_z = 0$ the vertical stationary position is identically found at $z^* = 0$ as it is illustrated in Figure \ref{Fig:geometry}.

The required vertical second phase derivative is given by $\phi^{P+G''}_{zz}  =  -\left( \frac{1}{\rho^G(\vx-\vxo)} + \frac{1}{\rho^P_v(\vxo)} \right)$, with $\rho^P_v$ and $\rho^G$ being the vertical radii of the target field and the Green's function (given identically by $\rho^G(\vx-\vxo) = |\vx-\vxo|$).
The above quantity frequently emerges in the following, therefore, for the sake of brevity we define the \emph{vertical/horizontal amplitude correction factor} as
\begin{equation}
  d^{\mathrm{c}}_{v/h}(\vx,\vxo)
    = \frac{\rho^{P}_{v/h}(\vxo) \rho^{G}(\vx- \vxo)}{\rho^{P}_{v/h}(\vxo) + \rho^{G}(\vx- \vxo)}.
\end{equation}

Approximating \eqref{eq:KA} by the leading order SPA given by Eq.~\eqref{eq:SPA0} results in the contour integral
{\fontsize{8pt}{\baselineskip}
\selectfont
\begin{multline}
  P(\vx,\omega) = \oint_C w(\vxo) \sqrt{\frac{8\pi}{\ti k}} 
  \sqrt{d^{\mathrm{c}}_{v}(\vx,\vxo)}
  P'_{\mathrm{n}}(\vxo, \omega)  G(\vx-\vxo,\omega)\td s(\vxo)
  \label{eq:25D_KA}
\end{multline}}
termed as the 2.5D Kirchhoff approximation.

Finally, it is feasible to apply the local plane wave approximation for the target field's gradient given by Eq.~\ref{eq:pw_appr}, a choice justified in the following section, resulting in the final form
{\fontsize{9pt}{\baselineskip}
\selectfont
\begin{multline}
  P(\vx,\omega)  \oint_C  w(\vxo)  \hkn^P(\vxo) \, \sqrt{8\pi \ti k} \, \sqrt{d^{\mathrm{c}}_{v}(\vx,\vxo)}\cdot
  \\ P(\vxo, \omega)  G(\vx-\vxo,\omega)\, \td s(\vxo).
  \label{eq:25D_KA}
\end{multline}}

As discussed in \cite{Firtha2025_ieee_a} in details the above 2.5D integral formulation can be evaluated asymptotically under high frequency conditions:
The result can be factorized to an intended component and additional diffractional waves due to the finite horizontal integral path.
The intended field can be estimated by evaluating Eq.~\eqref{eq:synth_field} around the integral's critical points, requiring the definition of a stationary position $\vxo^*(\vx)$ for each evaluation position $\vx$ (For the sake of brevity this dependency is not denotied in the following), resulting in
{\fontsize{8pt}{\baselineskip}
\selectfont
\begin{multline}
  P(\vx,\omega) \sim 
  4\pi w(\vxo^*) \, \sqrt{d^{\mathrm{c}}_{v}\left(\vx,\vxo^*\right)}   \sqrt{d^{\mathrm{c}}_{h}\left(\vx,\vxo^*\right)}
    P(\vxo^*, \omega) G(\vx-\vxo^*),
   \label{eq:25D_KA_amp_dist} 
\end{multline}}
coinciding with the target field.
This fact already shows that the validity of the 2.5D Kirchhoff approximation and the SPA are strongly interconnected: as long the SPA holds the KA inherently holds (asides from the undesired diffractional waves).

Furthermore, from \eqref{eq:WFS_amp_dist} the role of the amplitude correction factor is apparent:
In the frequency range where ray theory and the SPA holds, the amplitude distribution of the acoustic field is governed by the geometric spreading of energy, which is inherently defined by the phase function of the field, encoded in the wavefront curvature (as it is reflected by Eq.~(4) \cite{Firtha2025_ieee_a}).
Therefore, once the amplitude of the field is known at the stationary position and 3D propagation is assumed the amplitude is matched to the geometrically required $\frac{1}{\rho^{P}(\vxo) + \rho^{G}(\vx- \vxo)}$ by applying the correction factors $d^{\mathrm{c}}_{v}\cdot d^{\mathrm{c}}_{h}$, whereas $d^{\mathrm{c}}_{v/h}$ accounts for 2D propagation.

\subsection{The 2.5D WFS driving functions}
The 2.5D WFS driving function for an arbitrary SSD contour can be extracted from the 2.5D KA \eqref{eq:25D_KA} by making the kernel of the integral independent of the receiver position $\vx$, apparent in the amplitude correction factor.
This is achieved by realizing that under the validity of a further, horizontal SPA each SSD element at $\vxo$ dominates the synthesized field at those receiver positions for which the SSD element serves as a horizontal stationary position. 
This allows one to assign a unique reference position to each SSD position denoted by $\vxref(\vxo)$ as shown in Fig.~\ref{Fig:geometry}.
The set of all reference positions define the reference curve along which amplitude correct synthesis can be achieved. 
With these considerations the 2.5D WFS driving functions are given as 
{\fontsize{9pt}{\baselineskip}
\selectfont
\begin{multline}
  D(\vxo,\omega) = w(\vxo) \, \sqrt{\frac{8\pi} {\ti k}} \,  \sqrt{d^{\mathrm{c}}_{v}\left(\vxref(\vxo),\vxo\right)} \,  P'_{\mathrm{n}}(\vxo, \omega) \approx\\
  \approx w(\vxo)  \, \sqrt{8\pi \ti k} \, \sqrt{ d^{\mathrm{c}}_{v}\left(\vxref(\vxo),\vxo\right) } \,  \hkn^P(\vxo)\, P(\vxo, \omega).
  \label{eq:25D_WFS}
\end{multline}}
and similarly to \eqref{eq:25D_KA_amp_dist} the intended component of the synthesized field is given as
{\fontsize{8pt}{\baselineskip}
\selectfont
\begin{multline}
  \Psynth(\vx,\omega) \sim \\
  4\pi w(\vxo^*) \, \sqrt{ d^{\mathrm{c}}_{v}\left(\vxref(\vxo^*),\vxo^*\right)} \sqrt{ d^{\mathrm{c}}_{h}\left(\vx,\vxo^*\right)}
    P(\vxo^*, \omega) G(\vx-\vxo^*)
  =\\ =
    w(\vxo^*)
    \sqrt{\frac{ \rho^G(\vxref(\vxo^*)-\vxo^*) }{ \rho^P_v(\vxref(\vxo^*))} 
    \frac{ \rho^P_v(\vx)}{ \rho^G(\vx-\vxo^*) } }
    P(\vx,\omega).
   \label{eq:WFS_amp_dist} 
\end{multline}}
The above formulation trivially matches the target field at $\vx = \vxref$.

\section{Low frequency limits of WFS}

\begin{figure}[!t]
  \centering
  % LEFT COLUMN
  \begin{minipage}{0.49\columnwidth}
    \begin{overpic}[width=\linewidth]{figs/Fig_2_a.png}
      \footnotesize\put(0,2){(a)}
    \end{overpic}
  \end{minipage}
  \hfill
  % RIGHT COLUMN
  \begin{minipage}{0.49\columnwidth}
    \begin{minipage}{1\columnwidth}
      \begin{overpic}[width=\linewidth]{figs/Fig_2_b.png}
      \footnotesize\put(0,2){(b)}
    \end{overpic}
  \end{minipage}
    \vspace{0.1cm} % small vertical gap
    
    \begin{minipage}{1\columnwidth}
    \begin{overpic}[width=\linewidth]{figs/Fig_2_c.png}
      \footnotesize\put(0,2){(c)}
    \end{overpic}
  \end{minipage}
\end{minipage}

  \caption{Evaluation of a point source's field by the 2.5D appoximation at $f_0 = 1~\mathrm{kHz}$ (a), and the measured normalized freuqency response at the origin by applying the 3D (b) and the 2.5D Kirchhoff approximation.
  On Figure (a) white filled area denotes the amplitude distribution of the integrand.}
  \label{fig:Kirchhoff_appr}
\end{figure}

The main goal of the present contribution is to define a lower cutoff frequency above which WFS theory holds.
The previous section already highlighted that in the geometry under discussion---i.e. for sources and receivers located typically in the SSD's proximity ---this frequency coincides with the validity of the stationary phase approximation.

The validty of the stationary phase approximation is a complex problem that would require the careful investigation of the phase and amplitude relations of the entire integral path \cite{chassande-mottin_flandrin_1999, 721375}.
A local approximation on the low frequency limit can be given by investigating the KA's integrand's amplitude and phase changes around the stationary position, as discussed in \cite{firtha_hahn_schultz_fiala_2025} detailed in the appendix \ref{fiszfasz}.
This local approach, however, does not take the finite size of the SSD into account, which is shown to be dominating the low frequency limit in the aspect of WFS.

In the following a physically motivated approach is introduced, which takes into account finite size SSDs, however, does not investigate the relative rate of amplitude and phase changes in the integration path.
Therefore, the following approach is mainly applicable for smooth, enclosing SSDs, being a realistic requirement for WFS applications.
It should be noted that the following discussion both hold in the context of WFS and more generally for the 2.5D Kirchhoff approximation, therefore, they are discussed adjointly.

\subsection{Qualitative analysis}
For a qualitative analysis of the underlying principles, first the example of the 3D and 2.5D WFS is investigated, reproducing a point source inside a circular area, as illustrated in Fig.~\ref{fig:Kirchhoff_appr}.
Fig.~(a) illustrates the synthesized pressure and the solution of the equivalent scattering problem in the exterior domain.
Figures (b) and (c) depict the measured frequency response at the center of the circular SSD.
In order to investigate the structure of the synthesized field the gradient of the the target field given by Eq.~\eqref{eq:gradient} is applied to both the 3D and 2.5D WFS integrals allowing to partition them into a nearfield and farfield contribution.
Written for the 2.5D case this yiels 
{\fontsize{8pt}{\baselineskip}
\selectfont
\begin{multline}
  \Psynth(\vx,\omega)  = \oint_C w(\vxo) \, \sqrt{\frac{8\pi} {\ti k}} \, \sqrt{ d^{\mathrm{c}}_{v}\left(\vxref(\vxo),\vxo\right)} \left( \overline{\kappa}^P(\vxo) + \ti k\right) \cdot \\ \cdot 
  \, P'_{\mathrm{n}}(\vxo, \omega) G(\vx-\vxo,\omega) \td s(\vxo) 
  = I_{\mathrm{near}}(\vx,\omega) + I_{\mathrm{far}}(\vx,\omega) 
\end{multline}}
with
{\fontsize{9pt}{\baselineskip}
\selectfont
\begin{multline}
  I_{\mathrm{near}}(\vx,\omega) = \sqrt{\frac{8\pi}{\ti k}}\, \oint_C \overline{\kappa}^P(\vxo) w(\vxo)  \hkn^P(\vxo)   \, \sqrt{d^{\mathrm{c}}_{v}\left(\vxref(\vxo),\vxo\right)} \cdot
  \\ P(\vxo, \omega)  G(\vx-\vxo,\omega)\, \td s(\vxo), 
\end{multline}}
\vspace{-0.7cm}
{\fontsize{9pt}{\baselineskip}
\selectfont
\begin{multline}
  I_{\mathrm{far}}(\vx,\omega)  =  \sqrt{8\pi \ti k} \oint_C  w(\vxo)  \hkn^P(\vxo)  \,d^{\mathrm{c}}_{v}\left(\vxref(\vxo),\vxo\right) \cdot
  \\ P(\vxo, \omega)  G(\vx-\vxo,\omega)\, \td s(\vxo).
  \label{eq:Ifar}
\end{multline}}
and a similar decomposition can be written for the 3D case.
In Fig.~\ref{fig:Kirchhoff_appr} (b) and (c) all the results are normalized by the target field's nominal amplitude at the measurement position.

The amplitude distribution of the integrand in \eqref{eq:Ifar} is illustrated in Fig.~\ref{fig:Kirchhoff_appr} (a) over the SSD with white area.
Apparently, as even in case of an infinite planar/linear integration path the amplitude distribution decays rapidly towards the lateral integration positions, therefore, the amplitude distribution acts as a windowing function with an effective length defined in the following section \footnote{Similar effective aperture sizes can be defined for the 3D case as well. For the 3D case, however, the problem is more complex as in the present separable geometry a vertical and horizontal effective length can be defined and the behaviour of the integrals change at two cutoff frequencies, depending on these lengthes. 
This aspect is, however, out of the scope of the present discussion.}.
The contribution of $I_{\mathrm{near}}$ and $I_{\mathrm{far}}$ depend on how the phase function $\phi^P(\vxo) + \phi^G(\vx-\vxo)$ changes within this effective window length.
\begin{itemize}
  \item For low frequencies the phase barely changes over the effective window length, therefore, the integrand is approximately constant in phase over the SSD.
  As a result, for the 3D case the nearfield contribution is constant, while the farfield contribution exhibits a frequency response of $~\sim \ti k$.
  The constant nearfield contribution also shows a level decrease which is the error of the Kirchhoff approximation, highlighting that the KA is inherently unable to predict the target field's evanescent contribution/reactive near field.
  Obivously, this error vanishes in case of a planar integration path.
  \item For high frequencies where phase changes significantly within the effective window length. As the SPA starts to hold rapid oscillations cancel out and the nearfield contribution decreases with $~\sim \ti k$ while the farfield contribution is constant.
  \item In case of the 2.5D WFS the SPA introduces a frequency correction factor of $~\sim \sqrt{1/\ti k}$.
  For low frequencies, where the SPA does not hold this results in an erroneous low frequency boost for the near field given by $~\sim \sqrt{1 /\ti k}$, and results in an overall frequency response of $~\sim \sqrt{\ti k}$ for the farfield component.
  On the other hand when the SPA requirements is fulfilled $I_{\mathrm{far}}(\vx,\omega)$ exhibits the correct flat frequency response.
  This observation justifies the application of the hgih frequency gradient approximation in the 2.5D KA and 2.5D WFS theory.
\end{itemize}
From the above discussion it is clear that our aim is to find the amount of phase increment within the effective window length to find the low frequency limit of the 2.5D Kirchhoff approximation, and consequently of Wave Field Synthesis.

\subsection{Required phase increment on the SSD}

As a most simple estimation for the low frequency limit of WFS the cutoff frequency can be directly estimated from investigating the frequency response of the synthesized field, as shown in Fig.~\ref{fig:Kirchhoff_appr} (b), restricting our investigation to the farfield component $I_{\mathrm{far}}$.
It is known, that below the unknown cutoff wavenumber $k_{c}$ the slope of the synthesized field decreases by 10 dB/decade, i.e. with $\sim \sqrt{k}$.
The endpoint of the slope at $k = 0$ is given by 
\begin{multline} 
I_{\mathrm{far}}(\vx, 0) = \oint_C  w(\vxo)  \hkn^P(\vxo) \, \sqrt{8\pi} \, d^{\mathrm{c}}_{v}\left(\vxref(\vxo),\vxo\right)\cdot
  \\ A^P(\vxo)  A^G(\vx-\vxo)\, \td s(\vxo)
\end{multline} 
which coincides with the effective window length.
On the other hand applying piecewise linear approximation to the Bode plot it is assumed that synthesized field reaches it's ideal level at $\omega_c = c \cdot k_c$, which amplitude is given analytically by Eq.~\eqref{eq:WFS_amp_dist}, therefore,
\begin{multline} 
I_{\mathrm{far}}(\vx, \omega_{c}) = 4\pi w(\vxo^*) \, d^{\mathrm{c}}_{v}\left(\vxref(\vxo^*),\vxo^*\right) d^{\mathrm{c}}_{h}\left(\vx,\vxo^*\right) \cdot \\ \cdot 
    A^P(\vxo^*) A^G(\vx-\vxo^*)
\end{multline} 
holds.
From these considerations the cutoff frequency can be simply estimated as
\begin{equation}
  \small
 \sqrt{k_c(\vx)} I_{\mathrm{far}}(\vx, 0)  = I_{\mathrm{far}}(\vx, \omega_{c}) 
\hspace{2mm} \rightarrow \hspace{2mm}
  k_c(\vx) = \left(\frac{I_{\mathrm{far}}(\vx, \omega_{c})}{I_{\mathrm{far}}(\vx, 0)}\right)^2 
\label{eq:omega_c}
\end{equation}

The above formulation already allows one to estimate the lower cutoff frequency of 2.5D WFS (and of the 2.5D KA) by simple  geometrical considerations, however, it gives an accurate result only for frontal/on-axis positions (i.e. in front of a virtual sound source towards the center of the SSD) as it will be highlighted in later sections.
In order to arrive at a more general, physically motivated result it is investigated how the phase function of the integrand changes along the SSD at the critical cutoff frequency given by Eq.~\eqref{eq:omega_c}.

For the sake of simplicity consider the symmetric geometry depicted in Fig.xy:
The SSD is chosen to be a linear segment ...
Let's denote the center of the SSD, serving as a stationary position for $\vx$ as $\vxo^*$.
It is assumed that both the virtual source and the receiver lies sufficiently far from the integral path/SSD, thus the amplitude does not change over it significantly.
Yet, we allow the normal component of the local wavenumber vector and the tapering window to change considerably as they both vanish towards the SSD endpoints.
With these assumptions the endpoint of the slope at $k = 0$ is approximated as  
\begin{multline} 
I_{\mathrm{far}}(\vx, 0) \approx \sqrt{8\pi}  
d^{\mathrm{c}}_{v}\left(\vxref(\vxo^*),\vxo^*\right)
  \\ A^P(\vxo^*)\,  A^G(\vx-\vxo^*) \,\underbrace{\oint_C  w(\vxo) \,  \hkn^P(\vxo^*) \td s(\vxo)}_{\hat{L}_{\mathrm{eff}}}
\end{multline} 
with $\hat{L}_{\mathrm{eff}}$ being an equivalent, effective window length.
By substituting the above approximation to \eqref{eq:omega_c} cutoff wavenumber then can be approximated in the present setup by
%{\fontsize{7pt}{\baselineskip}
%\selectfont
%\begin{multline} 
%\omega_c(\vx) = 
%\frac{
 % 4\pi \sqrt{\frac{ \rho^P_v(\vxo^*) \, \rho^G(\vxref-\vxo^*) }{ \rho^P_v(\vxo^*) + \rho^G(\vxref-\vxo^*)} } 
%  \sqrt{\frac{\rho^P_h(\vxo^*) \, \rho^G(\vx-\vxo^*)}{  \rho^P_h(\vxo) + \rho^G(\vx-\vxo^*)}}
%   A^P(\vxo^*, \omega) A^G(\vx-\vxo^*) }{
%    \sqrt{8\pi} \sqrt{\frac{ \rho^P_v(\vxo^*) \, \rho^G(\vxref-\vxo^*) }{ \rho^P_v(\vxo^*) + \rho^G(\vxref-\vxo^*)} } 
%  A^P(\vxo^*)  A^G(\vx-\vxo^*) \hat{L}_{\mathrm{eff}}(\vx)
%   }
%\label{eq:omega_c_appr}
%\end{multline}}
{\fontsize{10pt}{\baselineskip}
\selectfont
\begin{equation} 
k_c(\vx) \approx
 2\pi 
 \frac{  d^{\mathrm{c}}_{h}\left(\vx,\vxo^*\right)^2 }{ 
   \hat{L}_{\mathrm{eff}}^2
   }
   =
    2\pi 
 \frac{  
  \frac{\rho^{P}_{h}(\vxo^*) \rho^{G}(\vx- \vxo^*)}{\rho^{P}_{h}(\vxo^*) + \rho^{G}(\vx- \vxo^*)}
 }{ 
   \hat{L}_{\mathrm{eff}}^2
   }
\label{eq:omega_c_appr}
\end{equation}}

In Appendix \ref{app:phase_increment} the phase increment from the stationary point to an arbitrary arc length is given by \eqref{eq:phase_inc}, reading as
{\fontsize{9pt}{\baselineskip}
\selectfont
\begin{equation} 
 \Delta \phi^{P+G}(\Delta s) =
- \frac{ \Delta s ^2}{2}  \frac{\hkn^P(\vxo^*)^2}{d^{\mathrm{c}}_{h}\left(\vx,\vxo^*\right)^2 }.
\end{equation}}
This allows one to express the total phase change up to the end of the equivalent window  $\Delta s = \hat{L}_{\mathrm{eff}} / 2$ on the cutoff wavenumber $k_c$ given by
{\fontsize{8pt}{\baselineskip}
\selectfont
\begin{equation} 
 k_c(\vx)\Delta \phi^{P+G}(\frac{\hat{L}_{\mathrm{eff}}}{2}) = -\frac{\pi}{4} \hkn^P(\vxo^*)^2
\end{equation}}
This result is fundemantel in the aspect of the present treatise: it states that in frontal directions ($\hkn^P(\vxo^*)=1$) the SPA starts to hold when the phase increment from the center of the array exceeds $\pi/4$ towards the end of the effective endpoint of the SSD, or with other word the effective window length contains at least a half wavelength.

The result is in perfect agreement with the theory of Fresnel zones and Fresnel numbers:
clearly, Eq.~\eqref{eq:omega_c_appr} can be reaaranged into a Fresnel number type quantity
 {\fontsize{10pt}{\baselineskip}
\selectfont
\begin{equation} 
1 \approx
 2\pi 
 \frac{  d^{\mathrm{c}}_{h}\left(\vx,\vxo^*\right)^2 }{ 
  k_c(\vx) \hat{L}_{\mathrm{eff}}^2
   }
\end{equation}}

\subsection{Low frequency limit for 2.5D WFS}

\section{Conclusion}

\appendix[Phase increment to arbitrary arc length]
\label{app:phase_increment}


Our aim is to estimate the phase increment along the SSD contour from the stationary position $\vxo^*$ (or $\vxo(s^*)$) up to an arbitrary arc length increment $\Delta s$.
As a first step the phase function can be expanded into a Taylor's series around the stationary point $\vxo^*$, reading 
{\fontsize{8pt}{\baselineskip}
\selectfont
\begin{equation}
  \phi^{P+G}(s) \approx \phi^{P+G}(s^*) + \left( s-s^* \right)\phi^{'P+G}_s(s^*) + \frac{\left( s-s^* \right)^2}{2}\phi^{''P+G}_{ss}(s^*)
  \label{eq:phase_taylors}
\end{equation}}
Note that for the sake of brevity notation $\vxo(s)$ is suppressed.
The phase increment than is expressed as  
{\fontsize{9pt}{\baselineskip}
\selectfont
\begin{multline} 
  \Delta \phi^{P+G}(\Delta s) =  \phi^{P+G}(s^* + \Delta s ) - \phi^{P+G}(s^*) = \\ =
   \Delta s \, \phi^{'P+G}_s(s^*) + \frac{ \Delta s^2}{2}\phi^{''P+G}_{ss}(s^*).
  \label{eq:phase_increment}
\end{multline}}

The first derivative $\phi^{'P+G}_s(s^*)$ read as
{\fontsize{8pt}{\baselineskip}
\selectfont
\begin{multline}
  \phi^{'P+G}_s(\vxo(s)) = \frac{\partial }{\partial s}\phi^{P+G}(\vxo(s)) = \\ =\nabla_{\vx}\phi^{P+G}(\vxo(s)) \cdot \frac{\partial \vxo(s)}{\partial s}
  = \hks^P(\vxo) - \hks^G(\vx- \vxo)
   \label{eq:first_deriv}
\end{multline}}
where $\frac{\partial \vxo(s)}{s}$ is the tangential vector of the SSD assuming an arc length parametrization.
For the second derivative the derivation is given in Appendix A of \cite{Firtha2025_ieee_a}, with the final result being
\begin{multline}
  \phi^{P+G''}_{ss}(\vxo(s)) = 
   -\hat{k}_{\mathrm{n}}(\vxo)^2 \left(\frac{1}{\rho^G_h(\vx-\vxo)} + \frac{1}{\rho^P_h(\vxo)} \right) - \\
   -2 \left( \hat{k}^P_{\mathrm{s}}(\vxo) - \hat{k}^G_{\mathrm{s}}(\vx-\vxo)\right) \kappa_{\mathrm{SSD}}(\vxo),
   \label{eq:curvature_horiz}
\end{multline}
where $\kappa_{\mathrm{SSD}}$ is the local curvature of the SSD.
By definition, in the stationary position $\hat{k}^P_{\mathrm{s}}(\vxo) = \hat{k}^G_{\mathrm{s}}(\vx-\vxo)$ holds, simplifying both \eqref{eq:first_deriv} and \eqref{eq:curvature_horiz} as
\begin{align}
   \phi^{'P+G}_s(\vxo(s^*)) &= 0,\\
    \phi^{P+G''}_{ss}(\vxo(s^*)) &= -\hat{k}_{\mathrm{n}}(\vxo)^2 \left(\frac{1}{\rho^G_h(\vx-\vxo)} + \frac{1}{\rho^P_h(\vxo)} \right) 
\end{align}
Finally, the required phase increment is given from \eqref{eq:phase_increment} by
{\fontsize{9pt}{\baselineskip}
\selectfont
\begin{multline} 
 \Delta \phi^{P+G}(\Delta s) = \frac{ \Delta s ^2}{2}\phi^{''P+G}_{ss}(s^*) = \\ =
- \frac{ \Delta s ^2}{2} \hat{k}_{\mathrm{n}}(\vxo^*)^2 \left(\frac{1}{\rho^G_h(\vx-\vxo^*)} + \frac{1}{\rho^P_h(\vxo^*)} \right).
\label{eq:phase_inc}
\end{multline}}

\section*{Acknowledgments}
This work was supported by the János Bolyai Research Scholarship of the Hungarian Academy of Science and by the OTKA PD-143129 and OTKA K-143436 grants.

%
\appendix[Phase increment at the SSD endpoint]
\paragraph{Virtual plane wave with no tapering applied}
In the following the special case of a virtual plane wave, synthesized by a finite SSD with the length of $L_x^{\mathrm{SSD}}$ is investigated.
The plane wave's propagation direction is perpendicular to the SSD ($k^P_x = 0, \, k^P_y = k$).
The synthesis is referenced to a reference line a along $\yref$, and the synthesized field is investigated in the frontal reference position $\vxref = [0,\, \yref,\, 0]$.
In this simple setup the synthesized field reads as 
\begin{equation}
  P(\vxref, \omega) = \int_{-L_x/2}^{L_x/2} \overbrace{\sqrt{ 8 \pi \ti k \yref }}^{D(\vxo, \omega)} \frac{\te^{- \ti k |\vxref - \vxo|}}{4 \pi |\vxref - \vxo|}  \td x_0,
\end{equation}
}
with $|\vxref - \vxo| = \sqrt{ x_0^2 + \yref^2 }$.
Since the amplitude of the target field at the reference position is unity, therefore, the cut-off wavenumber for the far-field approximation is simply given as
\begin{equation}
  k_{\mathrm{FF}} = \frac{1}{ \left( \int_{-L_x/2}^{L_x/2} \sqrt{ \frac{ \yref }{2 \pi}} \frac{1}{ |\vxref - \vxo|}  \td x_0 \right)^2}.
\end{equation}
For $\yref \gg L_x/2$ the amplitude changes only slightly over the integral path and the involved integral can be simply approximated as
\begin{equation}
  \small
  \int_{-L_x/2}^{L_x/2} \frac{1}{ \sqrt{x_0^2 + \yref^2}}  \td x_0 = \frac{1}{\yref} \int_{-L_x/2}^{L_x/2} \frac{1}{ \sqrt{ \underbrace{\left(\frac{x_0}{\yref}\right)^2}_{\ll 1} + 1}}  \td x_0 \approx \frac{L_x}{\yref},
\end{equation}
resulting in the cutoff wavenumber
\begin{equation}
  k_{\mathrm{FF}} \approx 2 \pi \frac{\yref}{ L_x^2}.
\end{equation}

Now we are interested about the phase variation of the integrand on this cutoff frequency.
In this frontal case the phase function is an even function of $x_0$, given merely by the phase of the Green's function $-k |\vxref - \vxo|$.
The phase increment measured from the center of the SSD to each end of the SSD at the cutoff wavenumber is given by
\begin{equation}
  \Delta \phi = k_{\mathrm{FF}} \left( \sqrt{\left(\frac{L_x}{2}\right)^2 + \yref^2} -  \yref \right).
  \label{eq:deltaPhi}
\end{equation}
Again, in case of $L_x \ll \yref$, distance of the reference point and the SSD endpoint can be expanded into a binomial (MacLaurin) series, which up to the leading order reads as
\begin{multline}
  \yref \sqrt{\left(\frac{L_x}{2 \yref}\right)^2 + 1} \approx \\ \approx \yref \left( 1 + \frac{1}{2} \left(\frac{L_x}{2 \yref}\right)^2 - \frac{1}{8}  \left(\frac{L_x}{2 \yref}\right)^4 + ...\right) = \\
  = \yref  + \frac{L_x^2}{8 \yref}. 
\end{multline}
Substituting back to \eqref{eq:deltaPhi} gives us the phase increment
\begin{equation}
  \Delta \phi = k_{\mathrm{FF}} \left( \frac{L_x^2}{8 \yref} \right) = 2 \pi \frac{\yref}{ L_x^2} \, \frac{L_x^2}{8 \yref} = \frac{\pi}{4}.
\end{equation}
This equation states that independently of the actual receiver position, in the far field of the SSD the phase change towards the SSD endpoint at the cutoff frequency/wavenumber is identically given by $\frac{\pi}{4}$.

\paragraph{Virtual plane wave with tapering}
As a more complicated extension let's assume an arbitrary tapering window applied to the driving functions $w(x_0)$, resulting in the synthesized field
\begin{equation}
  P(\vxref, \omega) = \int_{-L_x/2}^{L_x/2} \overbrace{ w(x_0)\sqrt{ 8 \pi \ti k \yref }}^{D(\vxo, \omega)} \frac{\te^{- \ti k |\vxref - \vxo|}}{4 \pi |\vxref - \vxo|}  \td x_0,
\end{equation}
Furthermore, let's introduce an effective SSD length defined as
\begin{equation}
  L_{x,\mathrm{eff}} = \int_{-\frac{Lx}{2}}^{\frac{L_x}{2}} w(x_0) \td x0!
\end{equation}
By following the same derivation as above the farfield cut-off frequency reads as
\begin{equation}
  k_{\mathrm{FF}} \approx 2 \pi \frac{\yref}{ L_{x,\mathrm{eff}}^2}.
\end{equation}
Finally, investigating the phase increment up to the point $L_{x,\mathrm{eff}}/2$ yields
\begin{equation}
  \Delta \phi = k_{\mathrm{FF}} \left( \sqrt{\left(\frac{L_{x,\mathrm{eff}}}{2}\right)^2 + \yref^2} -  \yref \right) \approx \frac{\pi}{4}.
\end{equation}
Hence, when a slowly changin tapering is applied to the driving functions, when the receiver is sufficiently far from the SSD, the phase change at the end of the effective length equals $\pi/4$ independently of the actual source position. 

\paragraph{Virtual point source}
Finally, assume a virtual point source located at $\vxs = [0,\, -\ys,\, 0]$.
The synthesized field with applying tapering, again, with referencing the synthesis to a parallel reference line is given by
\begin{multline}
  P(\vxref, \omega) = \sqrt{\ti k}  \frac{\ys}{4\pi\sqrt{2\pi}}\sqrt{  \frac{\yref}{\yref+\ys}} \\ \int_{-L_x/2}^{L_x/2} w(x_0)\frac{\te^{- \ti k |\vxo - \vxs|}}{ |\vxo-\vxs|^{3/2}}   \frac{\te^{- \ti k |\vxref - \vxo|}}{ |\vxref - \vxo|}  \td x_0,
\end{multline}
Now the amplitude of the integrand is approximated as 
\begin{multline}
  \small
  \int_{-L_x/2}^{L_x/2} \frac{w(x_0)}{ \sqrt{x_0^2 + \ys^2}^{3} \sqrt{x_0^2 + \yref^2}}  \td x_0 = \\
  \frac{1}{\ys^{3/2}\yref} \int_{-L_x/2}^{L_x/2} \frac{w(x_0)}{ \sqrt{ \left(\frac{x_0}{\ys}\right)^2 + 1}^{3/2} \sqrt{ \left(\frac{x_0}{\yref}\right)^2 + 1}}  \td x_0 \\ 
  \approx \frac{L_{x,\mathrm{eff}}}{\ys^{3/2}\yref},
\end{multline}
and the synthesized field at $\omega = 0$ is approximated as
\begin{equation}
  \scriptsize
  P(\vxref, 0) = \sqrt{\ti k}  \frac{L_{x,\mathrm{eff}}}{4\pi\sqrt{2\pi}}\sqrt{  \frac{1}{\ys\yref (\yref+\ys)}}.
\end{equation}
The target field in the present setup is given by
\begin{equation}
  P_{\mathrm{target}}(\vxref, 0) = \frac{1}{4\pi}\frac{1}{\yref+\ys},
\end{equation}
hence the synthesized field normalized by the target field is given by
\begin{equation}
  \scriptsize
  \hat{P}(\vxref, 0) = \sqrt{\ti k}  \frac{L_{x,\mathrm{eff}}}{\sqrt{2\pi}}\sqrt{  \frac{\yref+\ys}{\ys\yref }}.
\end{equation}
Again, the cutoff wavenumber is, therefore, approximated as
\begin{equation}
  k_{\mathrm{FF}} \approx \frac{2\pi}{L_{x,\mathrm{eff}}^2} \frac{\ys \, \yref}{\yref+\ys}.
  \label{eq:k_cutoff_ps}
\end{equation}

The phase change up to the endpoint of the SSD reads as
\begin{equation}
  \scriptsize
  \Delta \phi = k_{\mathrm{FF}} \left( \sqrt{\left(\frac{L_{x,\mathrm{eff}}}{2}\right)^2 + \yref^2} + \sqrt{\left(\frac{L_{x,\mathrm{eff}}}{2}\right)^2 + \ys^2} -  \yref - \ys \right).
\end{equation}
which by using its MacLaurin series can be written by 
\begin{equation}
  \Delta \phi = k_{\mathrm{FF}} \left( \frac{L_{x,\mathrm{eff}}^2}{8 \yref} + \frac{L_{x,\mathrm{eff}}^2}{8 \ys}  \right) =  k_{\mathrm{FF}} \frac{L_{x,\mathrm{eff}}^2}{8}\left(  \frac{\yref+\ys}{\yref \ys}  \right) 
\end{equation}
Finally, by substituting back \eqref{eq:k_cutoff_ps}
\begin{equation}
  \Delta \phi =  \frac{2\pi}{L_{x,\mathrm{eff}}^2} \frac{\ys \, \yref}{\yref+\ys} \frac{L_{x,\mathrm{eff}}^2}{8}\left(  \frac{\yref+\ys}{\yref \ys}  \right) = \frac{\pi}{4}.
\end{equation}
Again, it is verified analytically that in case of the synthesis of a virtual point source when the receiver and the virtual source is sufficiently far from the SSD (further than the effective SSD length) the cutoff frequency is given by that point where the phase change at the SSD's endpoint equals to $\pi/4$, i.e. the SSD contains the quater of the wavelength.

%\paragraph{Arbitrary virtual source}
\bibliographystyle{IEEEtran}
\bibliography{dissertation}

\end{document}


